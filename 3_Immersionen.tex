%!TEX root = main.tex
% vim: tw=0 noet sts=8 sw=8



%def von immersionen und formalen immersionen
% bemerkung über die topologie auf diesen räumen, einmal kompakt-offen topologie
% und das dies der C^1 topologie entspricht

\section{Immersionen}
\label{sec:Immersionen}
Neben der Spinstruktur ist es noch notwendig den Begriff der Immersion 
einzuführen. Dieser Begriff tritt auf natürliche Weise auf wenn man sich
mit Einbettungen von abstrakten Mannigfaltigkeiten in den $ \R^n $ beschäftigt.

\begin{Def}[Immersionen,formale Immersionen]
	\label{formIso}
	Eine Abbildung $\abb{f}{M}{N}$ zwischen zwei Mannigfaltigkeiten
	heißt \textit{Immersion}, falls das Differential $\abb{\mathrm{d} f}{TM}{TN}$ 
	injektiv ist.
	Ein Paar von Abbildungen $\abb{(f,F)}{M \times TM}{N \times TN} ,%
	(p,v) \mapsto (f(p),F(v))$ heißt \textit{formale Immersion} falls $ F $ faserweise linear und injektiv ist und folgendes
	Diagramm kommutiert:
	\begin{center}
		
		\begin{tikzcd}
			TM \arrow[rr, "F"] \arrow[dd, "\pi^M"'] & & TN \arrow[dd ,"\pi^N"] \\
			&& \\
			M \arrow[rr, "f"] & & N  
		\end{tikzcd}
		
	\end{center}
	
	Definiere nun $\Imm{M}{N},\Immf{M}{N}$ als Menge aller (formalen) Immersionen. Die Topologie für diese Mengen ist gegeben als Teilraumtopologie des Raumes $\map(M,N)$ der die Basis der Topologie 
	trägt mit folgenden Erzeugern\footnote{Dies ist nichts anderes als die kompakt-offen-Topologie}. Die Menge aller Mengen der Form $\set{f \in \map}{ f(K) \subset U}$
	für alle Paare $(K,U)$ von kompakten Teilmengen $K \subset M$ und
	offenen Teilmengen $U \subset N$ erzeugt die Topologie auf $\map(M,N)$.
\end{Def}

%bsp für formale immersion die keine immersion von der form (f,df) ist
%streckung?

Für das nächste Beipiel ist es folgender Fakt wichtig. Der Tangentialraum des $\R^n$ ist aufgrund der nachfolgenden Abbildung trivial:
\begin{align*}
\TT\R^n &\simeq \R^n \times \R^n \\
(p,v) &\mapsto (p,v)
\end{align*}
%Damit gilt nun folgendes Resultat

\begin{Bsp}
	\begin{enumerate}[\textbullet]
	\item Seien $ \abb{f,g}{\R^n}{\R^n} $ zwei beliebige glatte
	Abbildungen, wobei $ g $ linear und injektiv ist, dann gilt dass das Paar $ (f,(f,g)) $ eine formale
	Immersion ist. Diese formale Immersion wird genau dann von
	einer Immersion induziert falls $ g $ die Jacobimatrix von $ f $
	ist und $ f $ ein Immersion ist. Der genau Grund warum man hier eine sehr große Freiheit bezüglich der Wahl hat, ist dass das Tangentialbündel trivial ist. 
	\item foo \todo{Angabe einer formalen Immersion die nicht von einer Immersion induziert wird, für eine mfg deren tangbündel nicht trivial ist. zb S2}
	
	\item Betrachte die beiden Abbildungen 
	\begin{gather*}\label{immBsp}
		\abb{f}{\S^1}{\R^2}, \exp(i\phi)%
		\mapsto (\cos(\phi),\sin(\phi)),\\
		\abb{g}{\S^1}{\R^2}, \exp(i\phi)%
		\mapsto (\cos(2\phi),\sin(2\phi)),\\
		\abb{h}{\S^1}{\R^2}, \exp(i\phi)%
		\mapsto (\cos(2\phi),\sin(\phi)).
	\end{gather*}
	Diese sehen wie folgt aus.\\
\todo{bilder der beiden Immersionen}
	
	Die Abbildungen $f,h$ sind Immersionen, aber $g$ nicht wie sich leicht
	anhand der Definition erkennen lässt. Das interessante Beispiel ist
	das der Abbildung $h$, den diese ist nicht injektiv aber ihr Differential
	ist es.
	\end{enumerate}
\end{Bsp}

Für das nächste Theorem benötigen wir eine simple Aussage im $ \R^3 $.
\begin{Lem}\label{lem:hilfSmaleHirsch}
	Seien $ z_1,z_2\in\R^3 $ linear unabhängig. Dann gilt für alle
	$ u_1,u_2\in\R^3 $ mit $ \norm{u_1} < \frac{1}{2}\dist(z_1,\R(z_2)) $
	und $ \norm{u_2}<\frac{1}{2}\dist(z_2,\R(z_1)) $ das $ z_1+u_1,z_2+u_2 $
	linear unabhängig sind.
	
	Für $ v_1,v_2,v_3\in\R^3 $ mit $ v_1,v_2 $ linear unabhängig,
	$ v_3 \coloneqq \norm{v_1} \frac{v_1\times v_2}{\norm{v_1\times v_2}}$ 
	und $ \abb{W}{\R}{\R^2}, s\mapsto (\sin(2s),\sin(s)) $ gilt das ein $ k\in\R $
	mit $ \norm{e_1 + t \frac{\d W(s)}{\d s}} \geq k > 0 $ für alle $ t,s\in\R $
	existiert.
	
	Für die obigen $ v_1,v_2,v_3,k $ existiert ein $ \epsilon >0 $, sodass
	für alle $ v\in\Span(v_1,v_3) $ mit $ \norm{v} \geq k\norm{v_1} $ und $ \norm{u_1},\norm{u_2} <\epsilon $ gilt das $ v+z_1, v_2+u_2 $ linear
	unabhängig sind.
	\begin{proof}
		\todo{Bild einfügen}
		Die erste Aussage ist zu $ B_{1/2 \dist(z_1,\R z_2)}(z_1) \cap  B_{1/2\dist(z_2,\R z_1)}(z_2) = \emptyset $ äquivalent. Und mit
		\begin{align*}
			\norm{u_1+u_2} \leq 1/2 \dist(z_1,\R z_2) + 1/2 \dist(z_2,\R z_1)
			&\leq 1/2 \norm{z_1} + 1/2 \norm{z_2} \\
			&\leq \norm{z_1 - z_2} \leq \norm{z_1} + \norm{z_2}
		\end{align*}
		folgt die erste Behauptung.
		
		Die zweite Aussage ist klar, da die stetige Funktion $ (t,s) \mapsto \norm{e_1 + t \frac{\d W(s)}{\d s}} $ nach unten beschränkt ist und
		nicht verschwindet.
		\todo{letzte Aussage}
	\end{proof}
\end{Lem}

Wir wollen nun einen wichtigen Zusammenhang zeigen. 
%satz von smale hirsch als black box, aus intro to the h-principle
%finde gute referenz für dieses Theorem
\begin{Thm}[Smale-Hirsch-Theorem]\label{smalehirsch}
	Seien $M^m,N^n$ \mfgen, $m<n$ und M sei geschlossen. Dann ist die Abbildung
	$\abb{\theta}{\Imm{M}{N}}{\Immf{M}{N}}, f \mapsto (f,\mathrm{d} f)$ eine Homotopieäquivalenz.
	\begin{proof}
		Wir wollen dieses Theorem lediglich für den Fall $ M=M^2,N=\R^3 $
		zeigen, für den allgemeinen Fall sei auf \cite{SH} verwiesen.

		Wir werden in mehreren Schritten vorgehen. Zunächst wird die Aussage
		in einer Karte $ (U\subset M,(x,y)) $ und dann für einen 
		endlichen Atlas $ \mathcal{A} $ gezeigt.
		\begin{description}
			\item[In einer Karte $ U $:] Sei $ (\hat{b},b) \in \Immf{M}{\R^3}$ gegeben.
			Wir werden eine Homotopie 
			\begin{gather*}
			 \abb{\psi_t}{\Immf{M}{\R^3}}{\Immf{M}{\R^3}} 
			\end{gather*}
			  angeben mit $ \psi_0 = id $ und $ \psi_1(\theta(\Imm{M}{\R^3})) \subset \theta(\Imm{M}{\R^3}) $. Der Weg $ b_t \coloneqq \psi_t((\hat{b},b)) $ wird durch mehrere Teilwege zusammengesetzt.
			  
			  Wir benötigen für die Teilwege eine Funktion $ \abb{\phi_t}{U}{\R^3} $,$ t\in[0,1] $ mit den folgenden Eigenschaften:
			  \begin{enumerate}[\textbullet]
			  	\item $ \phi_0=0 $
			  	\item $b + \d \phi_t$ hat vollen Rang für alle $ t $.
			  	\item $ (1-t)b+t\d \hat{b} + \d \phi_1 $ hat vollen Rang für alle $ t\in[0,1] $.
			  \end{enumerate}
			  Angenommen wir hätten ein solche $ \phi_t $, dann lassen sich
			  die Teilwege auf die folgende Art und Weise angeben.
			  \begin{align*}
			  	(\hat{b} + \phi_t,b+\d \phi_t) & (\hat{b}+\phi_1,(1-t)b+t\d \hat{b} + \d \phi_1)
			  \end{align*}
			  Wobei die beiden Wege hintereinander geschaltet werden.
			  Aufgrund der Eigenschaften von dem gewählten $ \phi_t $ wäre dies
			  ein Weg in $ \Immf{M}{\R^3} $ der bei $ t=1 $ in $ \theta(\Imm{M}{\R^3}) $ bleibt. Die Glattheit der Wege ist klar
			  bis auf den Übergang bei der Verknüpfung der Teilwege. Doch lässt
			  sich dies abstrakt dahingehend beheben, das sich der Weg ausglätten
			  lässt, also beliebig genau durch einen glatten Weg approximieren 
			  lässt, wobei die Genauigkeit so gewählt wird, sodass der Weg
			  weiterhin in $ \Imm{M}{\R^3} $ liegt.
			  
			  Somit ist die Behauptung gezeigt falls man ein solches $ \phi_t $
			  findet. 
			  
			  \item[Konstruktion von $ \psi_t $:] Sei die Funktion $ W=(W_1,W_2) $ wie in \cref{lem:hilfSmaleHirsch} gegeben. Dann 
			  betrachten wir zu $ \abb{b}{U}{\R^{3\times 2}} , p\mapsto (v_1,v_2) \coloneqq b(p) $ wie in \cref{lem:hilfSmaleHirsch}
			  den dazugehörigen Vektor $ v_3 $ bilden daraus die Funktion
			  $ \abb{A}{U}{\GL_3^+} $. Mit $ \bar{W}\coloneqq (W_1,0,W_2)^t$ 
			  definieren wir die Funktion $ \varphi(x,y)\coloneqq aA(x,y)\bar{W}(cx) $ mit zwei fest gewählten Konstanten $ a,c\in\R $. Die Konstanten werden nun so gewählt das
			  $ b + t\d \varphi $ vollen Rang als Abbildung hat. Dies 
			  erreichen wir durch \cref{lem:hilfSmaleHirsch}, indem 
			  wir ein $ \epsilon > 0 $ wählen sodass $ \norm{a \frac{\partial A}{\partial x} \bar{W}(cx)},\norm{a \frac{\partial A}{\partial y}\bar{W}(cx)} < \epsilon $ gilt. Dann wissen wir duch \cref{lem:hilfSmaleHirsch} das in
			  \begin{align*}
			  	v_1 + t \frac{\partial \varphi}{\partial x} &= v_1 + ta\frac{\partial A}{\partial x} \bar{W}(cx) + tac\frac{\d W}{\d s}(cx) \\
			  	v_2 + t \frac{\partial \varphi}{\partial y} &= v_2 + ta \frac{\partial A}{\partial y} \bar{W}(cx)
			  \end{align*}
			  die Ausdrücke auf der rechten Seite die lineare Unabhängigkeit
			  nicht stören.
			  Diese Konstruktion wiederholt sich nun mit folgenden Änderungen.
			  Wir setzen $ \bar{W}=(0,W_1,W_2)^t $ und $ \norm{v_3}=\norm{v_2} $ und setzen schließlich $ \psi(x,y) = a^{'}A(x,y)\bar{W}(c^{'}y) $. 
			  Schlussendlich setzen wir als $ \phi_t $ die Hintereinanderausführung von $ t\varphi $ und $ \varphi + t\psi $. Damit ist die Abbildung $ \phi_t $ konstruiert.
			  
			\item[Im Atlas $ \mathcal{A} $:] \todo{für enen eindlichen Atlas zeigen}
			\end{description}
	\end{proof}
	
\end{Thm}

\begin{Bsp}
	Die Forderung $ \dim(M)<\dim(N) $ ist essentiell. Betrachte 
	den Fall $ M=N=\S^1 $ und die Abbildungen aus \cref{immBsp}:
	Dann induziert die Abbildung 
	\DefMap{\Imm{\S^1}{\S^1}}{\Immf{\S^1}{\S^1}}{f}{(f,\d f)}
	keine Homotopieäquivalenz, denn die Abbildung kann
	nicht alle Zusammenhangskomponenten von $ \Immf{\S^1}{\S^1} $
	treffen. Betrachte hierzu die konstante Abbildung
	\DefMap{f : \S^1}{\S^1}{z}{1} wobei wir $ \S^1\subset\c$ 
	verwenden, und den Diffeomorphismus $ \abb{\beta}{\S^1}{\R / \Z} $
	der durch die Winkelfunktion gegeben ist. Dann ist das Paar $ (f,\d \beta) $ eine formale Immersion, kann aber nicht regulär
	homotop zu einer Immersion sein, da Immersionen immer einen 
	Abbildungsgrad ungleich $ 0 $ haben, der von $ (f,\d \beta) $ aber
	verschwindet.
\end{Bsp}


%% Local Variables:
%% mode: latex
%% TeX-master: "main"
%% End:
