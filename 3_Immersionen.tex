%!TEX root = main.tex
% vim: tw=0 noet sts=8 sw=8



%def von immersionen und formalen immersionen
% bemerkung über die topologie auf diesen räumen, einmal kompakt-offen topologie
% und das dies der C^1 topologie entspricht

\section{Immersionen}

Neben der Spinstruktur ist es noch notwendig den Begriff der Immersion 
einzuführen.

\begin{Def}[Immersionen,formale Immersionen]
	\label{formIso}
	Eine Abbildung $\abb{f}{M}{N}$ zwischen zwei Mannigfaltigkeiten
	heißt \textit{Immersion}, falls das Differential $\abb{\mathrm{d} f}{TM}{TN}$ 
	injektiv ist.
	Ein Paar von Abbildungen $\abb{(f,F)}{M \times TM}{N \times TN} ,%
	(p,v) \mapsto (f(p),F(v))$ heißt \textit{formale Immersion} falls $ F $ faserweise linear und injektiv ist und folgendes
	Diagramm kommutiert:
	\begin{center}
		
		\begin{tikzcd}
			TM \arrow[rr, "F"] \arrow[dd, "\pi^M"'] & & TN \arrow[dd ,"\pi^N"] \\
			&& \\
			M \arrow[rr, "f"] & & N  
		\end{tikzcd}
		
	\end{center}
	
	Definiere nun $\Imm{M}{N},\Immf{M}{N}$ als Menge aller (formalen) Immersionen. Die Topologie für diese Mengen ist gegeben als Teilraumtopologie des Raumes $\map(M,N)$ der die Basis der Topologie 
	trägt mit folgenden Erzeugern\footnote{Dies ist nichts anderes als die kompakt-offen-Topologie}. Die Menge aller Mengen der Form $\set{f \in \map}{ f(K) \subset U}$
	für alle Paare $(K,U)$ von kompakten Teilmengen $K \subset M$ und
	offenen Teilmengen $U \subset N$ erzeugt die Topologie auf $\map(M,N)$.
\end{Def}

%bsp für formale immersion die keine immersion von der form (f,df) ist
%streckung?

Für das nächste Beipiel ist es folgender Fakt wichtig. Der Tangentialraum des $\R^n$ ist aufgrund der nachfolgenden Abbildung trivial:
\begin{align*}
\T \R^n &\simeq \R^n \times \R^n \\
(p,v) &\mapsto (p,v)
\end{align*}
%Damit gilt nun folgendes Resultat

\begin{Bsp}
	\begin{enumerate}[\textbullet]
	\item Seien $ \abb{f,g}{\R^n}{\R^n} $ zwei beliebige glatte
	Abbildungen, wobei $ g $ linear und injektiv ist, dann gilt dass das Paar $ (f,(f,g)) $ eine formale
	Immersion ist. Diese formale Immersion wird genau dann von
	einer Immersion induziert falls $ g $ die Jacobimatrix von $ f $
	ist und $ f $ ein Immersion ist. Der genau Grund warum man hier eine sehr große Freiheit bezüglich der Wahl hat, ist dass das Tangentialbündel trivial ist. 
	\item foo \todo{Angabe einer formalen Immersion die nicht von einer Immersion induziert wird, für eine mfg deren tangbündel nicht trivial ist. zb S2}
	
	\item Betrachte die beiden Abbildungen 
	\begin{gather*}\label{immBsp}
		\abb{f}{\S^1}{\R^2}, \exp(i\phi)%
		\mapsto (\cos(\phi),\sin(\phi)),\\
		\abb{g}{\S^1}{\R^2}, \exp(i\phi)%
		\mapsto (\cos(2\phi),\sin(2\phi)),\\
		\abb{h}{\S^1}{\R^2}, \exp(i\phi)%
		\mapsto (\cos(2\phi),\sin(\phi)).
	\end{gather*}
	Diese sehen wie folgt aus.\\
\todo{bilder der beiden Immersionen}
	
	Die Abbildungen $f,h$ sind Immersionen, aber $g$ nicht wie sich leicht
	anhand der Definition erkennen lässt. Das interessante Beispiel ist
	das der Abbildung $h$, den diese ist nicht injektiv aber ihr Differential
	ist es.
	\end{enumerate}
\end{Bsp}


%% Local Variables:
%% mode: latex
%% TeX-master: "main"
%% End:
