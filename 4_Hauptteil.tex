%!TEX root = main.tex
% vim: tw=0 noet sts=8 sw=8



\section{Zusammenhang von Immersionen und Spinstrukturen}
\label{sec:Zusammenhang von Immersionen und Spinstrukturen}

Im folgenden ist $(M,g)$ eine geschlossene, orientierte Mannigfaltigkeit.
In der Notation wird die Metrik $g$ unterdrückt.


Um einen Zusammenhang zwischen den Immersionen und den Spinstrukturen
herzustellen ist es notwendig den Raum der Immersionen zu
vereinfachen.  Wichtig ist hierbei das wir nur an den
Zusammenhangskomponenten interessiert sind, also können wir uns nicht
nur auf Bijektionen sondern auch auf homotopieäquivalente Räume
einschränken.  Wir nennen zwei Immersionen $ \abb{f,g}{M}{N} $
\textit{regulär homotop} falls sie sich in $ \Imm{M}{N} $ durch einen
glatten Weg verbinden lassen.

Im Weiteren betrachten wir den Spezialfall $ M=M^2 $ als geschlossene $ 2 $-dimensionale riemannsche Mannigfaltigkeit und
$ N=\R^3 $.

Das Smale-Hirsch Theorem \cref{smalehirsch} liefert uns eine erste Vereinfachung des Raumes. 

Der Raum von formalen Immersionen lässt sich noch weiter verkleinern
bzw. vereinfachen, indem wir anstatt von nur formalen Immersionen auch
diese betrachten bei denen die Abbildungen zwischen den
Tangentialräumen auch noch isometrisch ist, damit erhalten wir
folgenden


\begin{Satz}
  Seien die isometrischen, formalen Immersionen gegeben durch
  \begin{gather*}
    \Immfi{M}{\R^3} \coloneqq \set{(f,F) \in \Immf{M}{\R^3}}{F \text{
        ist isometrisch }}
  \end{gather*}
  Dann gilt das $\Immf{M}{\R^3}$ und $\Immfi{M}{\R^3}$
  homotopieäquivalent sind.
	\begin{proof}
          Betrachte für $t \in [0,1]$ die Abbildung
          $\abb{\psi_t}{\Immf{M}{\R^3}}{\Immf{M}{\R^3}}$ mit
          $\psi_t : (f,F) \mapsto (f,F \circ
          (F^{\ast}F)^{-t/2})$.
		
          Zunächst ist diese Homotopie eine glatte Abbildung für alle
          $ t \in [0,1] $. Für $ t=0 $ gilt $ \psi_0 = id $. Im Fall
          $ t=1 $ gilt nun das $ F \circ (F^{\ast}F)^{-t/2} $
          isometrisch ist.
		
          Der Ausdruck $ (F_p^\ast F_p)^{-t/2} $ ist wie folgt
          definiert. Sei $ \abb{f}{V}{V} $ eine linearer, injektiver,
          symmetrischer Endomorphismus eines reellen Vektorraums mit
          Skalarprodukt. Die Abbildung $ x\mapsto x^\alpha $ ist analytisch.
          Wir betrachten die Potenzreihenentwicklung $ x^\alpha = \sum_{k=1}^{\infty} a_k x^k $ und setzen dann $ f^\alpha \coloneqq \sum_{k=1}^{\infty} a_k f^k $. Die Konvergenz in $ V $ ist durch
          die Konvergenz der Potenzreihe garantiert.
%           dann existieren $ n $ positive, reelle
%          Eigenwerte $ \lambda_i $ für diesen Endomorphismus. Nach
%          Wahl einer Basis existiert eine unitäre Transformation
%          $ U $, sodass $ f = U^\ast \diag(\lambda_i) U $ gilt. Nun
%          definiere
%          $ f^\alpha \coloneqq U^\ast \diag(\lambda_i^\alpha) U $.
%          \todo{warum ist dies trotz der wahl wohldef?}
		
          Sei nun die lineare, injektive Abbildung
          $ \abb{F_p}{\tang{p}{M}}{\tang{F_p(p)}{\R^3}} $ gegeben,
          dann ist $ (F_p^\ast F_p) $ symmetrisch und aufgrund der
          Injektivität auch positiv definit, damit ist
%          $ (F_p^\ast F_p)^{-1/2} $ wohldefiniert.
		
          Es folgt nun durch Umformungen das die Abbildung
          $ F_p \circ (F_p^\ast F_p)^{-1/2} $ isometrisch
          ist. Sei hierzu $ v,w \in \tang{p}{M} $,so gilt
          \begin{align*}
            \scalar{F_p \circ (F_p^\ast F_p)^{-1/2}(v)}{F_p \circ (F_p^\ast F_p)^{-1/2}(w)}_{\R^3} &=
                                                                                                                     g_p(F_p^\ast F_p \circ (F_p^\ast F_p)^{-1/2}(v),(F_p^\ast F_p)^{-1/2}(w)) \\
                                                                                                                   &= g_p((F_p^\ast F_p)^1/2(v),(F_p^\ast F_p)^{-1/2}(w)) \\
                                                                                                                   &= g_p((F_p^\ast F_p)^{-1/2} \circ (F_p^\ast F_p)^{1/2}(v,w)) \\
                                                                                                                   &= g_p(v,w)
		\end{align*}
		
		Hierbei haben wir $ \scalar{v}{F_p(w)}_{\R^3} = g_p(F_p^\ast(v),w) $ für $ q =F_p(p) $ und $ v \in \tang{q}{\R^3},w\in \tang{p}{M} $ 
		benutzt.
		
		Dann liefert dies die gewünschte Homotopieäquivalenz.
	
	\end{proof}
\end{Satz}

Der nächste Schritt ist der des Vergessens des Basispunktes.

\begin{Lem}
  Es existiert eine Bijektion
	
%	\begin{align*}
%		\Immfi{M}{\R^3} &\rightarrow \set{ (\abb{F_p}{\T_pM}{\R^3})_{p\in M} }{\text{ isometrisch}}\\
%	\end{align*}.
  \begin{tikzcd}
    \Immfi{M}{\R^3} \arrow[r,"\eta"] &%
    \set{ (\abb{F_p}{\TT_pM}{\R^3})_{p\in M} }{ F_p \text{ isometrisch
        für alle }p \in M}.
  \end{tikzcd}
  \begin{proof}
    Definiere die Abbildung $ \eta $ wie folgt. Sei ein Paar
    $ (f,F) \in \Immfi{M}{\R^3}$ gegeben, dann setze als Zielobjekt
    schlichtweg die Familie $ (F_p)_{p\in M} $. Es ist noch zu zeigen
    das dies eine Bijektion ist.
    \begin{description}
    \item[Injektivität:] Seien zwei Paare $ (f,F),(g,G) $ mit
      $ \eta(f,F)=\eta(g,G) $ gegeben. Dann gilt nach Definition
      $ F_p = G_p $ für alle $ p\in M $. Also gilt $ F=G $ als
      Abbildungen. Da die Paare nun formale Immersionen sind gilt das
      die zugehörigen Diagramme aus \cref{formIso} kommutieren, also
      gilt:
      \begin{gather*}
        f \circ \pi^M = \pi_1 \circ F = \pi_1 \circ G = g \circ \pi^M
      \end{gather*}
      also
      \begin{gather*}
        f \circ \pi^M = g \circ \pi^M.
      \end{gather*}  Da die Projektion $ \pi^M $ surjektiv ist, folgt
      $ f=g $.  Also die Injektivität.
    \item[Surjektivität] Sei eine Familie $ (F_p)_{p\in M} $
      gegeben. Wähle eine beliebige glatte Funktion
      $ f\in \Cinfty{M}{\R^3} $. Setze nun als Urbild zu der anfangs
      gegebener Familie das Paar $ (f,F) $. Wobei $ F $ gegeben ist als
      \DefMap{\tang{}{M}}{\tang{}{\R^3}}{(p,v)}{(f(p),F_p(v)).}
			 
      Dann gilt offentsichtlich $ \eta(f,F)=(F_p)_{p\in M} $ und damit
      ist die Bijektivität nachgewiesen.
    \end{description}
  \end{proof}
\end{Lem}

Für das folgende Theorem benötigen wir die Existenz von Spinstrukturen
auf Flächen.
%lemma darüber das eine kompakte, orientierte 2dim riem mfg stets spin ist
\begin{Satz}\label{existenzspinflächen}
	Jede Fläche ist spin.
	\begin{proof}
		Wir haben schon in \cref{SpinstrSphäre} gezeigt das es 
		auf der $ \S^2 $ ein eindeutige Spinstruktur gibt. Wir
		wissen nun das sich alle Flächen als $ \S^2 $ mit Henkel 
		darstellen lassen \todo{bild}. Um nun die Spinstruktur auf der $ \S^2 $
		auf Flächen mit höheren Geschlecht zu erweitern, müssen
		wir zeigen das ausgehend von der $ \S^2 $ und zwei Punkten
		$ p,q \in\S^2 $ mit disjunkten Bällen $ B_p,B_q \subset\S^2$,
		durch zusammenkleben der beiden Umgebungen ein Henkel
		entsteht und die Spinstruktur erhalten bleibt.
		
		Diese Chirugie gestaltet sich wie folgt. Wir wählen auf
		$ [0,1] \times \S^1 $ eine der beiden möglichen Spinstrukturen \footnote{hieran erkennt man schon das man \textbf{nicht} alle $ 4 $ verschiedenen Spinstrukturen auf dem $ 2 $-Torus damit erhalten kann.} und betrachten 
		\begin{gather*}
		 \left( \S^2 \setminus (B_p \cup B_q) \amalg [0,1]\times \S^1 \right) / \sim = \TT^2.
		\end{gather*}
		Wobei $ \partial B_p \sim \{ 0 \}\times \S^1$ und $ \partial B_q \sim \{ 1 \}\times \S^1$ miteinanderverklebt werden, sodass die Orientierung
		erhalten bleibt.
		Wir müssen nun zeigen das für zwei spin Mannigfaltigkeiten bei 
		der disjunkten Summe ein eindeutige Spinstruktur existiert die 
		Erweiterungen der einzelnen Spinstrukturen ist.
	
		Hierzu betrachten wir zwei beliebige Flächen $ M,N $ und deren
		disjunkte Summe $ M \# N $. 	 \todo{ergänzen}
	\end{proof}
\end{Satz}
% satz das die imm und form-imm homotop sind

Sei $\S M$ definiert als $\set{(p,v) \in \TT M}{ \norm{v} = 1}$, wobei
die Norm gegeben ist als $ \norm{(p,v)} \coloneqq \sqrt{g_p(v,v)} $
und wir der einfachheithalber lediglich $ \norm{v} $ schreiben werden.

Für den nächsten Satz benötigen wir miteinander verträgliche $ \S^1 $-Gruppenwirkungen auf der $ \S M $ und $ \so_3 $. Es wird der
Zusammenhang $ \S^1 \simeq \so_2 $ vorausgesetzt. Für die erste
Gruppenwirkung wählen wir eine positiv orientierte Orthonormalbasis
$ (v,w) $ von $ \tang{p}{M} $ und betrachte den davon induzierten
Vektorraumisomorphismus $ \abb{\phi}{\R^2}{\tang{p}{M}},e_1 \mapsto v, e_2\mapsto w $. Diese
Abbildung liefert uns einen Liegruppenisomorphismus $\abb{\bar{\phi}}{\so(\R^2) = \so_2}{\so(\tang{p}{M})} $. Die Verkettung
dieser Isomorphismen liefert uns zu einem Element $ z\in\S^1 $ ein
eindeutiges Element $ U(z)\in \so(\tang{p}{M}) $. Damit ergibt sich
nun die Gruppenwirkung:
\DefMap{\S^1 \times \S_pM}{\S_pM}{(z,v)}{z*v\coloneqq U(z)v}

Für die zweite Gruppenwirkung benötigen wir den Fakt das sich jedes
Element $ A \in \so_3$  eindeutig in der Form $ A=\left( v~w~v\times w\right) $ mit $ v,w \in \R^3$ normiert und orthogonal zueinander, 
schreiben lässt. Durch die Wahl dieser Vektoren lässt sich auf
analoge Art wie oben ein Liegruppenisomorphismus $ \abb{\varphi}{\so_2}{\so(W)} $ finden, wobei $ W $ der von $ v,w $ 
im $ \R^3 $ aufgespannte Vektorraum ist. Damit operiert nun die
$ \S^1 $ auf dem $ \so_3 $.

Wir nennen nun eine glatte Abbildung $ \abb{H}{\S M}{\so_3}~\S^1$-invariant, falls $ H(z*v) = z*H(v) $ für alle $ z \in \S^1, v\in \S M $ gilt, also die Gruppenwirkung mit der Abbildung vertauscht. 
 

Der nächste Schritt wird der essentielle auf den Weg zu den
Zusammenhang mit den Spinstrukturen sein.

\begin{Satz}
  Es existiert eine Bijektion zwischen
  \begin{gather*}
    \set{ (\abb{F_p}{\TT_pM}{\R^3})_{p\in M} }{ F_p \text{ isometrisch
        für alle }p \in M}
  \end{gather*}
  und
  \begin{gather*}
    \set{ \abb{H}{\S M}{\so_3}}{H\text{ ist }\S^1 \text{-invariant}}.
  \end{gather*}
  \begin{proof}
    Sei eine Familie $(\abb{F_p}{\TT_pM}{\R^3})_{p\in M}$ von
    isometrischen Abbildungen gegeben.  Da es sich bei $M$ um eine
    zweidimensionale, orientierte riemannische Mannigfaltigkeit
    handelt, gilt das zu es jedem normierten Tangentialvektor
    $v \in \S_p M$ genau einen Tangentialvektor $J(v) \in \S_p M$
    existiert, sodass $(v,J(v))$ eine positiv, orientierte
    Orthonormalbasis für $\tang{p}{M}$ bildet. Betrachte nun die
    Abbildung $\abb{H_p}{\S_p%
      M}{\so_3}, v \mapsto (F_p(v),F_p(J(v)),F_p(v)\times F_p(J(v)))$.
    Diese Abbildung ist wohldefiniert aufgrund der Eigenschaften der
    Basis $(v,J(v))$ und dass das Kreuzprodukt der beiden
    Basiselemente in Matrixform ein Element der $\so_3$ liefert, hier nutzen wir das die Abbildugen $ F_p $ isometrisch sind.  Die
    $\S^1$ Invarianz dieser Abbildung folgt direkt aus den oben
    konstruierten Gruppenwirkungen.   
	
    Zeige nun das es sich bei dieser Abbildung um eine Bijektion
    handelt.
    \begin{description}
    \item[Injektivität:] Seien zwei Familien $(F_p)_p,(G_p)_p$ von
      isometrischen Abbildungen gegeben und sei
      $\vartheta((F_p)_p)=\vartheta((G_p)_p)$. Dann gilt aufgrund der
      Abbildungsvorschrift von $\vartheta$ das $F_p(v)=G_p(v)$ für
      alle $v \in \S_p M$ gilt und somit die Abbildungen gleich sind.
    \item[Surjektivität:] Sei eine $\S^1$-invariante Abbildung
      $\abb{H}{\S M}{\so_3}$ gegeben. Für einen normierten
      Tangentialvektor $v \in \S_p M$ gilt das $H_p(v)$ ein Element
      der $\so_3$ ist. Betrachte nun die erste Spalte der Matrix
      $w(v)=(H_p(v)_{i,1})_{i=1,2,3}$. Definiere nun eine Abbildung
      $\abb{F_p}{\tang{p}{M}}{\R^3}$ durch $v \mapsto \begin{cases}
        w(\frac{v}{\norm{v}}) & v \neq 0 \\
        0 & v=0.
      \end{cases}$
		
      Diese Abbildung liefert uns nun das gewünschte Urbild zur
      anfangs gewählten Abbildung $H$.
    \end{description}
    Der Nachweis der Bijektion liefert die Behauptung.
\end{proof}

\end{Satz}

Für den schlussendlichen Satz dieser Arbeit benötigen wir noch einige
Homotopiegruppen der $\so_3$ und $\spin_3$. Die verwendete Quelle ist
hierbei \cite{BHMMM15}.
\begin{Lem}\label{hgroups}
	Für die beiden Liegruppen $\spin_3$ und $\so_3$ gelten
	folgende Isomorphien als Mannigfaltigkeiten:
	\begin{align}
	\spin_3 &\simeq \su(2) \simeq \S^3 \label{id1}\\
	\so_3 &\simeq \RP^3 \label{id2}
	\end{align}
	Damit gelten auch folgende Isomorphien der Homotopiegruppen
	der beide Liegruppen.
	\begin{align*}
	\pi_k(\spin_3) &\simeq \begin{cases}
	1 & ,k=0,1,2 \\
	\Z & ,k=3
	\end{cases}
	&{\pi_k(\so_3) \simeq \begin{cases}
		1 & ,k=0,2 \\
		\Z_2 & ,k=1 \\
		\Z & ,k=3
		\end{cases}}
	\end{align*}
	
	\begin{proof}
		\begin{description}
                \item[1. Schritt:] Wir wollen zunächst die zweite
                  Identität \cref{id1} zeigen. Dieser Isomorphismus
                  ist direkt durch folgende Abbildung anzugeben
                  \DefMap{\set{\pma{a & -\bar{b} \\ b & \bar{a}} \in
                      \su_2}{a,b \in \c}}{\set{(a,b)\in
                      \c^2}{|a|^2+|b|^2=1}=\S^3}{\pma{a & -\bar{b} \\
                      b & \bar{a}}}{(a,b)}
			
                  Von \cref{id1} die erste Identität wird wie folgt
                  bewiesen.  Es gilt zunächst das die Spingruppe
                  $ \spin_3 $ eine Teilmenge der Cliffordalgebra
                  $ \Cl_3 $ ist und im geraden Teil dieser liegt, also
                  eine Untergruppe von $ \H^\ast $ ist. Man kann nun
                  zeigen das dies schon die gewünschte Behauptung
                  zeigt das $ \spin_3 \simeq \su_2 $ gilt.
				  Es folgt nun unmittelbar aufgrund des Isomorphismus
				  und den doppelten Überlagerung durch rausteilen
				  von $ \Z_2 $ die Isomorphie $ \RP^3\simeq\so_3 $.
			
                \item[2. Schritt;] Seien nun die obigen Identitäten
                  \cref{id1} gegeben, um nun die
                  die Homotopiegruppen zu berechnen reicht es aus
                  diese von $\RP^3$ und $ \S^3 $ zu kennen. Für
                  $ \S^3 $ sind diese bekannt und für $ \RP^3 $
                  stimmen diese für $ k \neq 1 $ mit denen von
                  $ \S^3 $ überein. Die Fundamentalgruppe von
                  $ \RP^3 $ ist bekannterweise $ \Z_2 $. Somit ist
                  alles gezeigt.
		\end{description}
		
		
		
	\end{proof}
	
\end{Lem}

Nach dieser weiteren Vereinfachung des Raums der Immersionen folgt nun
der letzte Schritt. Wir können nun folgenden Satz zeigen.



\begin{Satz}
	Es existiert eine Bijektion
	\begin{center}
		\begin{tikzcd}
			\set{ \abb{H}{\S M}{\so_3}}{H\text{ ist }\S^1 \text{-invariant}}/\sim  \arrow[rr,"\psi"] && \sset{\text{Spinstrukturen auf }M}.\\
		\end{tikzcd}
	\end{center}
	Wobei $ \sim $ hier bis auf Homotopie meint.
	\begin{proof}
          Im ersten Schritt des Beweises wählen wir eine
          Triangulierung $\Sigma \coloneqq \sset{ \Delta_i^j}$ der
          Mannigfaltigkeit. Wobei für den Simplex $\Delta_i^j$ der
          Index $i$ einer Durchnummerierung entspricht und der Index
          $j$ die Dimension des Simplex angibt.  \todo{bild von torus
            mit angedeuteter Triangulierung}
		
          Wir können zu auf folgende Art und Weise eine Abbildung
          $\psi$ konstruieren.  Zu einer $\S^1$-invarianten Abbildung
          $\abb{H}{\S M}{\so_3}$ definieren wir zunächst eine
          Spinstruktur auf dem $0$-Skelett und setzen diese induktiv
          auf die höheren Skelette fort um schließlich beim
          $2$-Skelett endend auf der gesamten Mannigfaltigkeit eine
          Spinstruktur zu erhalten. Schließlich zeigen wir noch das
          diese Konstruktion eine Bijektion liefert.
		
          \textbf{Konstruktion der Abbildung $\psi$}
		
          Sei nun eine $\S^1$-invariante Abbildung
          $\abb{H}{\S M}{\so_3}$ gegeben. Die Spinstruktur zu $ H $
          ist gegeben durch den Pullback des Diagramms $  $
              \begin{center}
              \begin{tikzcd}
                \Ph{\spin_2}{M} \arrow[r,"\phi"] \arrow[d] \arrow[dr, phantom, "\ulcorner", very near start] & \spin_3 \arrow[d,"2:1"] \\
                \Ph{\so_2}{M} = \S M \arrow[r,"H"] & \so_3
              \end{tikzcd}
            \end{center}
          Die Mannigfaltigkeit $ \Ph{\spin_2}{M} $ ist eine Spinstruktur
          auf $ M $. Diese ist eine zweifache Überlagerung des orientieren
          Rahmenbündels und hat eine $ \spin_2 $-Gruppenwirkung durch
          eine Wahl $ \spin_2\subset\spin_3 $ und den Morphismus $ \phi $.
          
          Wir müssen nun noch die Bijektivität bis auf Homotopie zeigen,
          d.h. wir bezüglich der gewählten Triangulierung lassen sich
          Injektivität und Surjektivität zeigen.
          
          \begin{description}
	          	\item[Injektivität:] Seien $ \abb{H,H^{'}}{\S M}{\so_3} $
	          	zwei $ \S^1 $-invariante Abbildungen mit $ \psi(H)=\psi(H^{'}) $. Wir wollen zeigen das die beiden
	          	Abbildungen homotop zueinander sind. Hierzu nutzen wir
	          	die Skelettstruktur der Mannigfaltigkeit aus.
		          	\begin{enumerate}[1{\bfseries -Skelett:}]
		          		\item[$ 0 ${\bfseries-Skelett:}] Für die endlich vielen $ 0 $-Skelette, also
		          		Punkte, auf der Mannigfaltigkeit lässt sich durch
		          		eine beliebige Verschiebung die Gleichheit von
		          		den Werten der beiden Abbildungen $ H,H^{'} $
		          		erreichen.
		          		\item Seien die Abbildungen auf dem $ 0 $-Skelett
		          		gleich, dann betrachten wir eine Kurve $ \gamma $
		          		die genau einen $ 1 $-Simplex durchläuft.
		          		Die Abbildung $ \abb{\dfrac{\dot{\gamma}}{\norm{\dot{\gamma}}}}{\S^1}{\S M = \Ph{\so}{M}} $ liftet genau dann auf $ \Ph{\spin}{M} $ wenn die Abbildungen
		          		$ H,H^{'} \circ \dfrac{\dot{\gamma}}{\norm{\dot{\gamma}}} $ auf
		          		$ \spin_3 $ liften.\todo{ergänzen}
		          		
			          	\item \OE~Seien die Abbildungen auf dem $ 1 $-Skelett
			          	gleich, 
		          	\end{enumerate}
          \end{description}
          
%          \begin{description}
%          \item[$0$-Skelett:] In diesem Fall ist die kanonische Wahl
%            zu treffen und es ist nichts zu zeigen.
%          \item[$1$-Skelett:] Für diesen Fall ist nun die Abbildung
%            $H$ zu benutzen. Wir wollen die Spinstruktur zunächst für
%            jeden $1$-Simplex der Triangulierung angeben.  Benutze
%            hierbei das ein $1$-Simplex $\Delta^1_i$ isomorph zur
%            $\S^1$ und damit auch zu $\S M$ ist. Wenn man nun
%            folgendes Diagramm betrachtet
%            \begin{center}
%              \begin{tikzcd}
%                && \spin_3 \arrow[dd,"2:1"] \\
%                &&\\
%                \Ph{\so_3}{\Delta^1_i} = \S M \arrow[rr,"H"] && \so_3
%              \end{tikzcd}
%            \end{center}
%            Wir betrachten nun den Pullback für dieses Diagramm und
%            erhalten somit das folgende kommutative Diagramm
%            \begin{center}
%              \begin{tikzcd}
%                \Ph{\spin_3}{\Delta^1_i} \arrow[rr] \arrow[dd] && \spin_3 \arrow[dd,"2:1"] \\
%                &&\\
%                \Ph{\so_3}{\Delta^1_i} = \S M \arrow[rr,"H"] && \so_3
%              \end{tikzcd}
%            \end{center}
%            erhalten damit eine Spinstruktur für den $1$-Simplex
%            $\Delta^1_i$. \todo{zeige das für zwei spinstruktuen auf
%              benachbarten 1-simplices die induzierten spinstrukturen
%              übereinstimmen}
%          \item[$2$-Skelett:] Es gilt nun noch zu zeigen das sich die
%            Spinstruktur auf den $1$-Skelett auf eindeutige Art und
%            Weise auf das $2$-Skelett fortsetzen lässt.  Sei nun
%            $\Delta^2_i$ gegeben und auf $\partial\Delta^2_i$ ist eine
%            Spinstruktur vorgegeben, dann existiert eine Fortsetzung
%            da nach \cref{hgroups} für die Homotopiegruppe der
%            $\spin_3$ gilt $\pi_k(\spin_3)=1$ für $k=1,2$.  Die
%            Tatsache für $k=1$ liefert uns die Existenz und $k=2$ die
%            Eindeutigkeit der fortgesetzten Spinstruktur.
%          \item[Injektivität:] \todo{warum ist mfg orientierbar falls 1skelett liftet, warum für spinstr bei 2skelett}
%          \item[Surjektivität:]\todo{spinstr ist orientierung des loopspace von M}
%          \end{description}
	\end{proof}
\end{Satz}



%% Local Variables:
%% mode: latex
%% TeX-master: "main"
%% End:
