% encodings                                        
\usepackage[T1]{fontenc}                            
\usepackage[utf8]{inputenc}                         
\usepackage[shorthands=off]{babel}                  
\usepackage{lmodern}                                
\usepackage{microtype}                              
\usepackage{scrlayer-scrpage}                       
\usepackage{enumerate}                              
%\usepackage{enumitem}
\usepackage{csquotes}   
\usepackage{slashed}				%Dirac slash
\usepackage{todonotes}     
%\usepackage{lscape}
% for \FloatBarrier
\usepackage{placeins}   
%	\usepackage{fancyhdr}				%includes a fancy header on each page

                 
%\usepackage[backend=biber,style=numeric-comp]{biblatex}
%\bibliography{literatur.bib}
\usepackage[backend=bibtex, style=numeric-comp, sorting=nty]{biblatex}
\bibliography{literatur}

% for fancy colors
\usepackage{xcolor}


% %maths                                            
\usepackage{amsmath, amsthm, amssymb, bm, bbm, mathtools, dsfont, mathrsfs}

%zum spielen
\usepackage{verbatim}

%commutative diagramms
\usepackage{tikz}
\usepackage{tikz-cd}
\usetikzlibrary{babel}
% links, TODO
%\usepackage[pdftitle={Seminar: "K1 von Ringen"},pdfauthor={Julian Seipel}, pdfsubject={Seminararbeit}]{hyperref}
%TODO: abschluss-projekt von tex-kurs heranziehen

%fancyheader
%\usepackage{fancyhdr, blindtext}

%\renewenvironment{proof}{{\bfseries Proof}}{*something*}
% \newenvironment{myproof}[1][\proofname]{%
%   \proof[\ttfamily \scshape \large #1 (yes, ``#1'')]%
% }{\endproof}

%must be loaded as lastest
\usepackage{cleveref}
% out of lemma 1.1 to Lemma 1.1
\let\cref=\Cref                                                    
%special mathsymbols                                

%title stuff
%	\let\oldtitle\title
%	\renewcommand{\title}[1]{\oldtitle{#1}\newcommand{\mythetitle}{#1}}
%	\let\oldauthor\author
%	\renewcommand{\author}[1]{\oldauthor{#1}\newcommand{\mytheauthor}{#1}}	

%% fancy plain style
%	\fancypagestyle{plain}{%
%		\fancyhf{}%
%		\fancyfoot[C]{}%
%		\renewcommand{\headrulewidth}{0pt}% Line at the header invisible
%		\renewcommand{\footrulewidth}{0pt}% Line at the footer invisible
%	}	

\newcommand{\C}{\mathds{C}}                         
\newcommand{\R}{\mathds{R}}                         
\newcommand{\N}{\mathds{N}}                         
\newcommand{\Z}{\mathds{Z}}                         
\newcommand{\Q}{\mathds{Q}}

% oder alternativ, funktioniert aber nicht
%\defmathbbsymbols{Z Q C}
%\defmathbbsymbolsubs{N R}


\newcommand{\pma}[1]{\begin{pmatrix} #1 \end{pmatrix}}
\newcommand{\A}{\mathscr{A}}
\newcommand{\F}{\mathbb{F}}                                            
\newcommand{\gs}{$\{a_0,\ldots ,a_n\}$ }
\newcommand{\gr}[1]{\left| #1 \right|}
% TODO: schreibe newcommand, das #2 als optional nimmt, und somit kein
% \Big| setzt
\newcommand{\set}[2]{\bigl\{ #1 \;  | \; #2 \bigr\}}
\newcommand{\sset}[1]{\bigl\{ #1 \bigr\}}
\renewcommand{\sp}[2]{\sum\limits_{i=#1}^{#2}}

\newcommand{\nn}[1][\,\bullet\,]{\left\| #1 \right\|}	% norm       
\newcommand{\foo}[2][]{#1 und #2 }

\newcommand{\ch}{\conv(\sset{a_0,\ldots , a_n})}

%\newcommand{\SS}{\mathbb{S}}
                    
%%%%%%%%%%%%%%%%%%%%                                
%theoremstyle                                       
\newtheoremstyle{mythms}
 {15pt}% space above
 {12pt}% space below 
 {\mdseries}% body font
 {}% indent amount
 {\bfseries}% theorem head font
 {:}% punctuation after theorem head
 {0.6cm plus 0.25cm minus 0.1cm}% space after theorem head (\newline possible)
 {}% theorem head spec 


                          
% math (theorems)                     
%\theoremstyle{Definition}             
%\newtheorem{Def}{Definition}[section] 
\theoremstyle{mythms}             
\newtheorem{Def}{Definition}[section] 
\newtheorem{Bsp}[Def]{Beispiel}       

%\theoremstyle{remark}               
\newtheorem{Bem}[Def]{Bemerkung}      
\newtheorem{Wdh}[Def]{Wiederholung}   

%\theoremstyle{plain}                  
\newtheorem{Satz}[Def]{Satz}     
\newtheorem{Thm}[Def]{Theorem}     
\newtheorem*{satz}{Satz}              
\newtheorem*{Beh}{Behauptung}
\newtheorem{Lem}[Def]{Lemma}          
\newtheorem{Kor}[Def]{Korollar}       
\newtheorem{Fol}[Def]{Folgerung}      

%a colletion of theorem env die {} a only used for folding up
{
%%%%%%%%%%%%%%%%%%%%%%%%%%%%%%%%%%%%%%%%%%
%     loeh
%	\newtheorem{thm}{Theorem}[section]
%	\newtheorem{prop}[thm]{Proposition}
%	\newtheorem{lem}[thm]{Lemma}
%	\newtheorem{cor}[thm]{Corollary}
%	\newtheorem{conj}[thm]{Conjecture}
%	\newtheorem{problem}[thm]{Problem}
%	\newtheorem{question}[thm]{Question}
%	\newtheorem{setup}[thm]{Setup}
%	
%	\theoremstyle{remark}
%	\newtheorem{rem}[thm]{Remark}
%	\newtheorem{exa}[thm]{Example}
%	\newtheorem{nonexa}[thm]{(Non-)Example}
%	\newtheorem{notation}[thm]{Notation}
%	\newtheorem*{ack}{Acknowledgements}
%	
%	\theoremstyle{definition}
%	\newtheorem{defi}[thm]{Definition}          
%%%%%%%%%%%%%%%%%%%%%%%%%%%%%%%%%%%%%%%%%%

% pilca

%\theoremstyle{plain}
%
%\newtheorem{theorem}{Theorem}[section]
%%\newtheorem{theorem}[theorem]{Theorem}
%\newtheorem{proposition}[theorem]{Proposition}
%\newtheorem{lemma}[theorem]{Lemma}
%\newtheorem{corollary}[theorem]{Corollary}
%\newtheorem{conjecture}[theorem]{Conjecture}
%\newtheorem{principle}[theorem]{Principle}
%\newtheorem{claim}[theorem]{Claim}
%
%\theoremstyle{definition}
%
%%\newtheorem{roughdef}[subsection]{Rough Definition}
%\newtheorem{definition}[theorem]{Definition}
%\newtheorem{remark}[theorem]{Remark}
%\newtheorem{remarks}[theorem]{Remarks}
%\newtheorem{example}[theorem]{Example}
%\newtheorem{examples}[theorem]{Examples}
%%\newtheorem{problem}[subsection]{Problem}
%%\newtheorem{question}[subsection]{Question}

% strohmaier

% \newtheorem{theorem}{Theorem}[section]
% \newtheorem{lem}[theorem]{Lemma}
% \newtheorem{cor}[theorem]{Corollary}
% \newtheorem{pro}[theorem]{Proposition}
% \newtheorem{hyp}[theorem]{Hypothesis}
% \newtheorem*{maintheorem}{Main Theorem}
% 
% \theoremstyle{definition}
% \newtheorem{definition}[theorem]{Definition}
% \newtheorem{exa}[theorem]{Example}
% \newtheorem{rem}[theorem]{Remark}
% \newtheorem{rems}[theorem]{Remarks}
% \newtheorem{exercise}[theorem]{Exercise}

%	nikolai

%	\declaretheoremstyle[
%	spaceabove=6pt, spacebelow=6pt,
%	headfont=\normalfont\bfseries,
%	notefont=\bfseries, notebraces={(}{)},
%	bodyfont=\normalfont,
%	postheadspace=1em,
%	qed=$\lozenge$
%	]{mystylesingle}
%	
%	\declaretheorem[style=mystylesingle,name=Problem]{Prb}
%	\declaretheorem[style=mystylesingle,name=Main Theorem]{MainThm}
%	\declaretheorem[style=mystylesingle,name=Question]{Qst}
%	
%	\declaretheoremstyle[
%	spaceabove=6pt, spacebelow=6pt,
%	headfont=\normalfont\bfseries,
%	notefont=\bfseries, notebraces={(}{)},
%	bodyfont=\normalfont,
%	postheadspace=1em,
%	qed=$\lozenge$,
%	]{mythmstyle}
%	
%	
%	\declaretheoremstyle[
%	spaceabove=6pt, spacebelow=6pt,
%	headfont=\normalfont\bfseries,
%	notefont=\mdseries, notebraces={(}{)},
%	bodyfont=\normalfont,
%	postheadspace=1em,
%	qed=$\square$,
%	]{myprfstyle}
%	
%	
%	\declaretheorem[style=mythmstyle,name=Definition,numberwithin=section]{Def}
%	
%	\declaretheorem[style=mythmstyle,sibling=Def,name=Theorem]{Thm}
%	\declaretheorem[style=mythmstyle,sibling=Def,name=Example]{Exm}
%	\declaretheorem[style=mythmstyle,sibling=Def,name=Convention]{Con}
%	\declaretheorem[style=mythmstyle,sibling=Def,name=Corollary]{Cor}
%	\declaretheorem[style=mythmstyle,sibling=Def,name=Fact]{Fct}
%	\declaretheorem[style=mythmstyle,sibling=Def,name=Lemma]{Lem}
%	\declaretheorem[style=mythmstyle,sibling=Def]{NumText}
%	\declaretheorem[style=mythmstyle,sibling=Def,name=Remark]{Rem}
%	\declaretheorem[style=mythmstyle,sibling=Def,name=Conjecture]{Cnj}
%	\declaretheorem[style=mythmstyle,sibling=Def,name=Notation]{Not}
%	\declaretheorem[style=mythmstyle,sibling=Def]{recall}	
%	
%	\declaretheorem[style=myprfstyle,numbered=no,name=Proof]{Prf}	
%	
%	
%	%Specify plural forms for all cleverrefs for the new theorems
%	\crefname{Def}{Definition}{Definitions}
%	\crefname{Exm}{Example}{Examples}
%	\crefname{Con}{Convention}{Conventions}
%	\crefname{Cor}{Corollary}{Corollaries}
%	\crefname{Fct}{Fact}{Facts}
%	\crefname{Lem}{Lemma}{Lemmas}
%	\crefname{NumText}{Paragraph}{Paragraphs}
%	\crefname{Rem}{Remark}{Remarks}
%	\crefname{Cnj}{Conjecture}{Conjectures}
%	\crefname{Exc}{Exercise}{Exercises}
%	\crefname{MainThm}{Main Theorem}{Main Theorems}
%	\crefname{Prb}{Problem}{Problems}
%	\crefname{Qst}{Question}{Questions}
%	\crefname{Thm}{Theorem}{Theorems}
%	\crefname{Not}{Notation}{Notation}
%	\crefname{Prf}{Proof}{Proofs}
%	

	
	
	}




                                                  
%own color for proof environment
\definecolor{mygray}{rgb}{0.2,0.2,0.2}

%\renewenvironment{center}{\begin{quote}\bfseries}{\end{quote}}
% use for font, [slshape,mdseries]

\renewenvironment{proof}{{ \vspace{0.5cm} \bfseries ~\newline Beweis:\newline}\slshape\color{mygray}}{\hfill$\blacksquare$}


%\renewenvironment{proof}{{ \vspace{0.5cm} \bfseries ~\newline Pr$\infty$f:\newline}\slshape\color{mygray}}{\hfill$\blacksquare$}


%TODO: schrift soll geändert werden,
%\renewenvironment*{proof}{\bfseries}{}

%stuff used only a few times                        
\DeclareMathOperator{\D}{D}
\DeclareMathOperator{\pow}{P} 
\DeclareMathOperator{\f}{\varphi} 
\DeclareMathOperator{\GL}{GL}
\DeclareMathOperator{\SL}{SL}
\DeclareMathOperator{\K}{K_1}
\DeclareMathOperator{\SK}{SK_1}
\DeclareMathOperator{\E}{E}
\DeclareMathOperator{\Mat}{Mat}
\DeclareMathOperator{\diag}{diag}
\DeclareMathOperator{\Int}{Int}
\DeclareMathOperator{\B}{B}
\DeclareMathOperator{\Bn}{\overline{\B_{n}(0)}}
%\DeclareMathOperator{\Sp}{S} 
%better
	\renewcommand{\S}{\operatorname{S}}	
	\renewcommand{\d}{\operatorname{d}}	%or use \mathrm{d}
\DeclareMathOperator{\V}{V}
\DeclareMathOperator{\conv}{conv}
\DeclareMathOperator{\imm}{Imm}
\DeclareMathOperator{\immf}{\Imm^{form}}
% spinOp and soOP are the Operators for the principial bundles
\DeclareMathOperator{\spinOp}{\sf Spin}
\DeclareMathOperator{\soOp}{\sf SO}
\DeclareMathOperator{\su}{\sf SU}
\DeclareMathOperator{\map}{Abb}
\DeclareMathOperator{\T}{T}
\DeclareMathOperator{\Ad}{Ad}
\DeclareMathOperator{\Adt}{\widetilde{\operatorname{Ad}}} % Ad twisted
\DeclareMathOperator{\id}{id}

%better epsilon
\def\epsilon{\varepsilon}

                      
             	                  

%%%%%%%%%%%%%%%%%%%%%%%%%%%%%%%%%%%%%%%%%
% from foreign sources
{

%\newcommand*{\largecdot}{\raisebox{-0.25ex}{\scalebox{1.2}{$\cdot$}}}


%%%%%%%%%%%% interesting manipulation
%\makeatletter% Damit man Footnotes ohne Nummer machen kann.
%\def\blfootnote{\gdef\@thefnmark{}\@footnotetext}
%\makeatother

% what does providecommand
%\providecommand{\bysame}{\leavevmode\hbox to3em{\hrulefill}\thinspace}
%\providecommand{\MR}{\relax\ifhmode\unskip\space\fi MR }

%%%%%%%%%%%%%%%%%%%%%%
%collection of all mathical formation
%\begin{gather*}
%\mathbf{ABCDEFGHIJKLMNOPQRSTUVWXYZ}\\
%\mathbb{ABCDEFGHIJKLMNOPQRSTUVWXYZ}\\
%\mathcal{ABCDEFGHIJKLMNOPQRSTUVWXYZ}\\
%\mathds{ABCDEFGHIJKLMNOPQRSTUVWXYZ}\\
%\mathfrak{ABCDEFGHIJKLMNOPQRSTUVWXYZ}\\
%\mathit{ABCDEFGHIJKLMNOPQRSTUVWXYZ}\\
%\mathrm{ABCDEFGHIJKLMNOPQRSTUVWXYZ}\\
%\mathsf{ABCDEFGHIJKLMNOPQRSTUVWXYZ}\\
%\mathtt{ABCDEFGHIJKLMNOPQRSTUVWXYZ}\\ 
%\end{gather*}


%\newcommand{\mnote}[1]{\marginpar{\tiny\em #1}} 

% fancy sum macro
%\newcommand{\suma}[2]{\overset{\left[\frac{#2}{2}\right]}{\underset{\ell=#1}{\sum}}}
	
% nice macros
%\newcommand{\FE}{F\hspace{-0.8mm}E}
%\newcommand{\gc}{\check g}	
	

%%	these two macros does the same thing
%\def\Ric{{\mathrm{Ric}}}
%\DeclareMathOperator{\Ric}{Ric}



%%	equivalent macros
%	\newcommand{\definedas}{\mathrel{\raise.095ex\hbox{\rm:}\mkern-5.2mu=}}
%	\coloneqq
%




%\DeclareMathOperator{\Cst}{{\sf C}\,\text{\raisebox{-.1em}{${}^*$}}}
%\DeclareMathOperator{\Set}{{\sf Set}}
%\DeclareMathOperator{\Sett}{{Set}}
%\DeclareMathOperator{\Top}{{\sf Top}}
%\DeclareMathOperator{\Toppt}{{\sf Top}_*}
%\DeclareMathOperator{\Toppth}{{\sf Top}_{* \sf h}}
%\DeclareMathOperator{\Toph}{{\sf Top}_{\sf h}}
%\DeclareMathOperator{\Group}{{\sf Group}}
%\DeclareMathOperator{\Vect}{{\sf Vect}}
%\DeclareMathOperator{\sn}{{\sf sn}}
%\DeclareMathOperator{\Ab}{{\sf Ab}}
%\DeclareMathOperator{\CW}{{\sf CW}}
%\DeclareMathOperator{\CWh}{{\sf CW}_{{\sf h}}}
%\DeclareMathOperator{\CWpt}{{\sf CW}_*}
%\DeclareMathOperator{\snAb}{{\sf snAb}}
%\DeclareMathOperator{\snAbfin}{{\sf snAb}^{{\sf fin}}}
%\DeclareMathOperator{\Ch}{{\sf Ch}}
%\DeclareMathOperator{\CoCh}{{\sf CoCh}}
%%TODO: what are this def are doing
%\def\LCh#1{{}_{#1}\!\Ch}
%\def\RCh#1{\Ch_{#1}}
%%%
%\DeclareMathOperator{\Ob}{Ob}
%\DeclareMathOperator{\Mor}{Mor}
}
%%%%%%%%%%%%%%%%%%%%%%%%%%%%%%%%%%%%%%%%%


\DeclarePairedDelimiter{\norm}{\lVert}{\rVert}

\newcommand{\LLeftrightarrow}{~\Leftrightarrow ~}
\let\Leftrightarrow=\LLeftrightarrow

%loop operator
\newcommand{\mmm}{\looparrowright}

%\renewcommand{\theenumi}{\alpha{enumi}}
%\renewcommand{\thepage}{\Roman{\page}}
%\addtokomafont{part}{\itshape}
%\setkomafont{section}{\normalfont\Large}
%\setkomafont{subsection}{\normalfont\Large}
%\pagenumbering{Roman}

%\DeclareMathOperator{\E}{E}
% \newcommand{\exxp}[1]{\exp \left( #1 \right)}       
% \renewcommand{\sin}[1]{\sin \left( #1 \right)}      

%stuff, sometimes needed

%TIP: vermeide $$-env in newcommands
\newcommand{\RMF}{riemannische Mannigfaltigkeit }
\newcommand{\RMFen}{riemannischen Mannigfaltigkeiten }
\newcommand{\mfg}{Mannigfaltigkeit }
\newcommand{\mfgen}{Mannigfaltigkeiten }
\newcommand{\Ph}[2]{P_{#1}#2}
\newcommand{\spin}[1]{\spinOp_{ #1 }}
\newcommand{\so}[1]{\soOp_{ #1 } }
\newcommand{\scalar}[2]{\langle #1 {,} #2 \rangle}
\newcommand{\scdots}{\scalar{\cdot}{\cdot}}%{$\langle\cdot{,}\cdot\rangle$}
\newcommand{\CP}{\C P}
\newcommand{\RP}{\R P}
\newcommand{\abb}[3]{#1 \colon #2 \rightarrow #3} % f\colon D \to Z$
\newcommand{\Imm}[2]{\imm(#1,#2)}
\newcommand{\Immf}[2]{\imm^{form}(#1,#2)}
\newcommand{\Immfi}[2]{\imm^{form}_{iso}(#1,#2)}
%\newcommand{\Gl}[1]{\GL(#1)}
\newcommand{\tang}[2]{\T_{#1}#2}

\newcommand*{\pd}[2][]{\frac{\partial #1}{\partial #2}}

	%cat and fun
	\newcommand{\cat}[1]{\mathbf{#1}}
	\newcommand{\fun}[1]{\bm{\mathrm{#1}}}

	%DefMap
	\newcommand{\DefMap}[4]{
		\begin{align*}
		\begin{array}{rcl}
		#1 & \to & #2 \\
		#3 & \mapsto & #4 
		\end{array} 
		\end{align*}
	}


%\wr liefert für diagramme eine tilde<

%%% Local Variables:
%%% mode: latex
%%% TeX-master: "main"
%%% End: