%!TEX root = main.tex
% vim: tw=0 noet sts=8 sw=8



\begin{Satz}
	Es existiert eine Bijektion
	\begin{center}
		\begin{tikzcd}
			\set{ \abb{H}{\S M}{\so{3}}}{\S^1 \text{-invariant}}  \arrow[rr,"\psi"] && \sset{\text{Spinstrukturen auf }M}.\\
		\end{tikzcd}
	\end{center}
\begin{proof}
	Im ersten Schritt des Beweises wählen wir eine Triangulierung $\Sigma \coloneqq \sset{ \Delta_i^j}$ der Mannigfaltigkeit. Wobei für den Simplex $\Delta_i^j$ der Index $i$  
	einer Durchnummerierung entspricht und der Index $j$ die Dimension des Simplex angibt.
	\todo{bild von torus mit angedeuteter Triangulierung}
	
	Wir können zu auf folgende Art und Weise eine Abbildung $\psi$ konstruieren.
	Zu einer $\S^1$-invarianten Abbildung $\abb{H}{\S M}{\so{3}}$ definieren wir
	zunächst eine Spinstruktur auf dem $0$-Skelett und setzen diese induktiv
	auf die höheren Skelette fort um schließlich beim $2$-Skelett endend auf
	der gesamten Mannigfaltigkeit eine Spinstruktur zu erhalten. Schließlich
	zeigen wir noch das diese Konstruktion eine Bijektion liefert.
	
	\textbf{Konstruktion der Abbildung $\psi$}
	
	Sei nun eine $\S^1$-invarianten Abbildung $\abb{H}{\S M}{\so{3}}$ gegeben.
	\begin{description}
		\item[$0$-Skelett:] In diesem Fall ist die kanonische Wahl zu treffen und es
			ist nichts zu zeigen.
		\item[$1$-Skelett:] Für diesen Fall ist nun die Abbildung $H$ zu benutzen. Wir wollen die Spinstruktur zunächst
		für jeden $1$-Simplex der Triangulierung angeben. 
		Benutze hierbei das ein $1$-Simplex $\Delta^1_i$ isomorph
		zur $\S^1$ und damit auch zu $\S M$ ist. Wenn man nun
		folgendes Diagramm betrachtet
		
		\begin{center}\begin{tikzcd}
			&& \spin{3} \arrow[dd,"2:1"] \\
			&&\\
			\Ph{\so{3}}{\Delta^1_i} = \S M \arrow[rr,"H"] && \so{3}
		\end{tikzcd}\end{center}
			Wir betrachten nun den Pullback für dieses Diagramm
			und erhalten somit das folgende kommutative Diagramm
		\begin{center}\begin{tikzcd}
			\Ph{\spin{3}}{\Delta^1_i} \arrow[rr] \arrow[dd] && \spin{3} \arrow[dd,"2:1"] \\
			&&\\
			\Ph{\so{3}}{\Delta^1_i} = \S M \arrow[rr,"H"] && \so{3}
		\end{tikzcd}\end{center}
	 erhalten damit eine Spinstruktur für den $1$-Simplex
		$\Delta^1_i$. \todo{zeige das für zwei spinstruktuen auf benachbarten 1-simplices die induzierten spinstrukturen übereinstimmen}
		\item[$2$-Skelett:] Es gilt nun noch zu zeigen das sich
		die Spinstruktur auf den $1$-Skelett auf eindeutige Art
		und Weise auf das $2$-Skelett fortsetzen lässt.
		Sei nun $\Delta^2_i$ gegeben und auf $\partial\Delta^2_i$
		ist eine Spinstruktur vorgegeben, dann existiert eine 
		Fortsetzung da nach \cref{hgroups} für die Homotopiegruppe
		der $\spin{3}$ gilt $\pi_k(\spin{3})=1$ für $k=1,2$.
		Die Tatsache für $k=1$ liefert uns die Existenz und $k=2$
		die Eindeutigkeit der fortgesetzten Spinstruktur.
		\item[Injektivität:] 
	\end{description}
\end{proof}
\end{Satz}




%%%% Local Variables:
%%%% mode: latex
%%%% TeX-master: "main"
%%%% End:
