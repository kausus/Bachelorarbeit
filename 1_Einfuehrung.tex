%kapitel Einführung und hinführung zum thema, erklärung von konventionen
\section*{Einführung}

In dieser Bachelorarbeit soll ein nichttrivialer Zusammenhang zwischen
dem Raum aller Immersionen einer zweidimensionalen, kompakten, orientierten Mannigfaltigkeit und der Menge aller Spinstrukturen (bis auf Isomorphie) auf dieser Mannigfaltigkeit
bewiesen werden.



Im \autoref{sec:Hauptfaserbündel und Spinstruktur} der Arbeit werden die notwendigen Begriffe wie Hauptfaserbündel und Spinstruktur eingeführt und anhand von Beispielen vertieft.
Im \autoref{sec:Immersionen} werden die Begriffe Immersion und formale Immersion eingeführt und das wichtige Smale-Hirsch-Theorem
bewiesen. Schließlich wird in \autoref{sec:Zusammenhang von Immersionen und Spinstrukturen} das Hauptresultat bewiesen.
