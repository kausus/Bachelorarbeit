% themenbereiche
%Geometrische simpliziale Komplexe; Triangulierungen; abstrakte
%simpliziale Komplexe; Beispiele simplizialer Komplexe


%homotopiegruppen sind invarianten von top räumen, einfaches kriterium
%zur unterscheidung von top räumen




% schlagwort verzeichnis
%TODO: folgende begriffe müssen definiert, verwendet und verstanden sein: simplex, seite, eckmenge, dimension, simplizialkomplex, n-skelett, geometrischer simplizialkomplex, geometrische realisierung, schwache topologie, triangulierbar, simpliziale abbildung, baryzentrische koordinaten, 

%warum simpliziale komplexe bzw mengen gebraucht werden, selbe homotophietheorie und selbe homotopiegruppen, einfachere berechnung der homotopiegruppen.
%
\section{Hauptfaserbündel und Spinstruktur}

In diesem ersten Abschnitt soll die grundlegenden Strukturen eingeführt 
werden. Der Begriff des Hauptfaserbündel hat sich als der Richtige
herausgebildet indem die Sprache von Spinstrukturen bzw. Spinoren
formuliert wird.

Die verwendete Quelle ist \cite{baum09}

\begin{Def}[Hauptfaserbündel]
	Ein \textsf{Hauptfaserbündel} $(P,\pi,M,G)$ ist ein tupel das
	aus folgenden Daten besteht
	\begin{enumerate}
		\item $P,M$ sind Mannigfaltigkeiten und $\pi : P \rightarrow M$ ist eine lokale Trivialiserung, d.h. 
		%TODO: definition vervollständigen und zwei beispiel anfügen
	\end{enumerate}		
\end{Def}

\begin{Bsp}
	blub%TODO: zweifache Überlagerung der S^1 und Rahmenbündel, beides wird später bei der erklärung der spinstruktur gebraucht, die zweifache Überlagerung als nichtriviales beispiel einer spinstruktur und rahmenbündel aufgrund des spinorbündels bzw der assoziiertenbündel konstruktion
\end{Bsp}


Im folgenden wird der Begriff der Spinstruktur benötigt, es 
hierzu lediglich die Defintion und zwei Beispiele angegeben.
Die verwendete Quelle hierzu ist \cite{BHMMM15}.

%TODO: ist es klar dass das paar (\phi...) eine mfg ist? soll das diagramm größer gemacht werden?
\begin{Def}[Spinstruktur]
	Sei $(M,g)$ eine \RMF der Dimension $m$, dann heißt
	 ($\Ph{\spin{n}}{M}$,$\eta : \Ph{\spin{n}}{M} \rightarrow \Ph{\so{n}}{M}$) eine Spinstruktur auf $(M,g)$, falls folgndes gilt:
	 \begin{itemize}
	 	\item Die \mfg $\Ph{\spin{n}}{M}$ ist ein $\spin{n}$ Hauptfaserbündel.
	 	\item Die Abbildung $\eta$ ist eine doppelte Überlagerung.
	 	\item Folgendes Diagramm kommutiert\\
		 	\begin{center}
		 		% das zeichen ' nach "\chi" ändert die position bzgl pfeil
		 	\begin{tikzcd}
		 		\spin{n} \ar[rr, "a \mapsto ua"] \ar[dd, "\chi"'] & &  \Ph{\spin{n}}{M}  \arrow[rd] \arrow[dd, "\eta"] & \\
		 		&&& M \\
		 		\so{n} \arrow[rr , "A \mapsto \eta(u)A"] & & 
		 		\Ph{\so{n}}{M} \arrow[ur] &
		 	\end{tikzcd}
			 \end{center}
			wobei $\chi$ die doppelte Überlagerungsabbildung von $\so{n}$ ist.
	 \end{itemize}
	
	

\end{Def}


%
%
%Topologische Räume sind im allgemeinen sehr schwer zu beschreibende
%mathematische Objekte. Dieses Seminar behandelt eine Methode, eine
%bestimmte Klasse von topologischen Räumen durch einfache geometrische
%Objekte, den Simplizies, zu beschreiben und zu verstehen.
%
%Simplexe sind einfache kombinatorische Objekte, mit denen sich denen
%sich eine große Klasse von topologischen Räumen beschreiben lassen.
%Diese bilden eine Verallgemeinerung von Punkt, Gerade, gleichseitigem Dreieck
%und dem Tetraeder. Diese werden auf höhere Dimensionen verallgemeinert.
%
%Wir definieren zunächst grundlegende Begriffe für die anschließende
%Defintion der Simplexe.
%
%\begin{Def}[Geometrisch unabhängig\footnote{Oder auch affine
%    Unabhängigkeit}]
%  \label{def:1}
%  Eine endliche Menge $\{ a_0,\ldots ,a_n \} \subset \R^N$ heißt
%  \textbf{geometrisch unabhängig}, falls das System von Vektoren
%  \begin{gather*}
%    a_0 - a_1 , a_0 - a_2 \ldots , a_0 - a_n
%  \end{gather*}
%  linear unabhängig im Sinne der Linearen Algebra ist. Für eine
%  Indexmenge $J$ heißt eine beliebige Menge $A \subset \R^J$
%  geometrisch unabhängig, falls jede endliche Teilmenge von $A$
%  geometrisch unabhängig im obigen Sinne ist.
%\end{Def}
%Wir zeigen für spätere Verwendung eine äquivalente Formulierung der
%geometrischen Unabhängigkeit.
%
%\begin{Lem}\label{lem:1}
%  Teilsysteme von geometrisch unabhängigen Systemen sind geometrisch
%  unabhängig. Eine endliche, geometrisch unabhängige Menge des $\R^N$
%  hat maximal $N+1$ Elemente. Für eine Menge
%  $\{ a_0 , \ldots , a_n \} \subset \R^N$ und Parametern
%  $(t_1,\ldots ,t_n) \in [0,1]^n$ sind folgende Aussagen äquivalent:
%
%  \begin{enumerate}[(i)]
%  \item Das System $\{ a_0 , \ldots , a_n \}$ ist geometrisch
%    unabhängig.
%  \item Für $\sum\limits_{i=0}^n t_i = 0$ und
%    $\sum\limits_{i=0}^n t_i a_i = 0$ folgt stets $t_i = 0$ für alle
%    $i \in \{ 0,\ldots,n\}$.
%  \end{enumerate}
%\end{Lem}
%\begin{proof}
%    Sei $\{ a_0 , \ldots , a_n \}$ ein geometrisch unabhängiges System
%    und $\{ a_{i_0},\ldots,a_{i_r} \}$ hierzu ein Teilsystem.  Sei
%    \OE~ $i_0 = 0$, sonst nummeriere um oder betrachte ein anderes
%    $i_j$, so dass für dieses $i_j$ das ursprüngliche geometrisch
%    unabhängige System der Punkt $a_{i_j}$ als Basispunkt gewählt
%    werden kann. Nun ist nach \cref{def:1},
%    $ a_0 - a_1 , \ldots , a_0 - a_n$ linear unabhängig und somit auch
%    das Teilsystem $ a_0 - a_ {i_1}, \ldots , a_0 - a_{i_r}$.  Sei nun
%    $A \subset \R^N$ ein endliches, geometrisch unabhängiges System,
%    dann ist nach \cref{def:1}, für ein Element $a \in A$, das System
%    $\bigl\{ a - a \Big| a' \in A \setminus \sset{a} \bigr\}$
%    linear unabhängig.
%    Die Kardinalität des Systems ist offentsichtlich durch die
%    Dimension des Vektorraum beschränkt und somit maximal gleich $N$,
%    mit dem Basispunkt also $N+1$. Für den letzten Teil des Beweises
%    nutze man folgende Äquivalenzen
%
%    \begin{description}
%    \item[i) $\Rightarrow$ ii)] Seien $\sp{0}{n} t_i = 0$ und $\sp{0}{n} t_i
%      a_i = 0$, dann folgt 
%      \begin{gather*}
%        0 = \sp{0}{n} t_i a_i = t_0 a_0 + \sp{1}{n} t_i a_i =
%        \sp{1}{n} (-a_0 t_i) + \sp{1}{n} t_i a_i = \sp{1}{n} t_i (a_i -
%        a_0)
%    \end{gather*}
%    und somit $t_i = 0$ für alle $i$.
%      \item[ii) $\Rightarrow$ i)] Betrachte folgende Gleichungen
%        \begin{gather*}
%          0 = \sp{1}{n} t_i (a_i - a_0) = \sp{1}{n} (-a_0 t_i) +
%          \sp{1}{n} t_i a_i = \sp{0}{n} t_i a_i.
%        \end{gather*}
%        Mit $t_0 \coloneqq - \sp{1}{n} t_i$ folgt nun mit ii) die
%        Behauptung.\qedhere foo
%    \end{description}
%  \end{proof}
%%\end{Lem}
%
%In der nächsten Definition wird das in diesem Vortrag zugrundeliegende
%Objekt von Interesse definiert. Dieses dient im weiteren Verlauf als
%Baustein für die Komplexe.
%
%\begin{Def}[Geometrischer $n$-Simplex]
%  Zu einem geometrisch unabhängigen System \gs $\subset \R^N$, nennt
%  man die Menge
%  \begin{gather*}
%    \set{\sum\limits_{i=0}^n t_i a_i \in \R^N}{ t_i \in [0,1] ~,~
%      \sum\limits_{i=0}^n t_i = 1}
%  \end{gather*}
%  den (geometrischen) $n$-Simplex und schreibt $a_0 \ldots a_n$ oder
%  ohne die Punkte genauer zu spezifizieren $\sigma^n$. Dies ist die
%  Menge aller Konvexkombinationen des Systems \\\gs. Als Konvention ist
%  der Simplex stets in den $\R^N$ eingebettet für ein $N \geq n$.
%\end{Def}
%
%
%\begin{Bsp}[Einfache Beispiele]
%  Definiere für $n \in \N$ den \textbf{Standardsimplex}
%  $\sigma^n \coloneqq e_0\ldots e_n$. Die Fälle $n = 0,1,2,3$ sind
%  hier aufgezeichnet. Beachte das mit $e_0 = 0$ dies nicht der
%  Konvention aus der Vorlesung entspricht.
%  
%% 0 - simplex, punkt
%% besseren punkt ausdenken, kleinen kreis mit blauer farbe, 
%\begin{tikzpicture}
%  \node (s) at (0,2) {$n=0$};
%  \fill (0,0) circle[radius=0.07cm];
%\end{tikzpicture}
%\hfill
%%1 - simplex, gerade
%\begin{tikzpicture}
% \node (s) at (0,2) {$n=1$};
%  \fill (-0.7,0) circle[radius=0.07cm];
%%  \draw[thin] (0,0)--++(1,0);
%  \fill (0.7,0) circle[radius=0.07cm];
%
%  \draw [thin,fill=lightgray] (-0.7,0) to (0.7,0) to (-0.7,0);
%\end{tikzpicture}
%\hfill
%%2 - simplex, gleichseitiges dreieck
%%PROBLEM: drucker druckt nicht den fill
%\begin{tikzpicture}
%    \node (s) at (0,1.5) {$n=2$};
%  \draw [thin,fill=lightgray] (90:1) to (210:1) to (330:1) to (90:1);
%    \fill (90:1) circle[radius=0.07cm];
%    \fill (210:1) circle[radius=0.07cm];
%    \fill (330:1) circle[radius=0.07cm];
%\end{tikzpicture}
%\hfill
%%3 - simplex, tetraeder
%\begin{tikzpicture}
%  \node (s) at (1,2) {$n=3$};
%\draw [thin,fill=lightgray] (0,0.2) to (1.8,0) to (1,1.7) to (0,0.2);
%\draw [thin,fill=lightgray] (1.8,0) to (1.8,0.7) to (1,1.7) to (1.8,0);
%\draw [thin,dashed] (1.8,0.7) to (0,0.2);
%
%    \fill (0,0.2) circle[radius=0.07cm];
%    \fill (1.8,0) circle[radius=0.07cm];
%    \fill (1,1.7) circle[radius=0.07cm];
%    \fill (1.8,0.7) circle[radius=0.07cm];
%\end{tikzpicture}
%
%\end{Bsp}
%
%Um Punkte aus einem Simplex unabhängig von der Lage des Simplex im
%umliegenden Raum zu beschreiben, sind spezielle Koordinaten
%vonnöten. Diese werden nun definiert und ihre Eindeutigkeit und
%Stetigkeit bezüglich des Punktes bewiesen.
%
%
%\begin{Lem}[Baryzentrische Koordinaten]\label{lem:bary}
%  \normalfont Es bezeichnet $x$ einen Punkt aus dem Simplex
%  $\sigma^n = a_0\ldots a_n$ und $(t_1,\ldots ,t_n) \in [0,1]^n$. In
%  der Darstellung $x = \sum_{i=0}^n t_i a_i$ nennt man $t_i$
%  \textbf{baryzentrische Koordinaten}. Diese sind durch $x$ eindeutig
%  bestimmt und als Funktionen
%  $t_i : \sigma^n \subset \R^N \rightarrow [0,1]$ stetig.
%  \begin{proof}
%    Zeige zunächst die Eindeutigkeit über die äquivalente
%    Formulierung der geometrischen Unabhängigkeit aus \cref{lem:1}
%    \begin{description}
%    \item[Eindeutigkeit: ] Seien zwei Darstellungen
%      \begin{gather*}
%        x = \sum\limits_{i=0}^n t_i a_i = \sum\limits_{i=0}^n s_i a_i
%        \text{ mit } \sum\limits_{i=0}^n t_i = \sum\limits_{i=0}^n s_i
%        = 1 \text{ und } t_i,s_i \in [0,1]
%      \end{gather*}
%      gegeben, so folgt durch umformen
%      \begin{gather*}
%        \sum\limits_{i=0}^n (t_i - s_i ) \cdot a_i = 0 \text{ und }
%        \sum\limits_{i=0}^n t_i - s_i = 0.
%      \end{gather*}
%      Da $\sset{a_0 , \ldots , a_n}$ ein geometrisch unabhängiges
%      System ist, folgt durch \cref{lem:1}, dass $t_i - s_i = 0 $ und
%      damit $ t_i = s_i$. Also sind die Koordinaten eindeutig.
%    \item[Stetigkeit: ] Aufgrund der Eindeutigkeit der baryzentrischen
%      Koordinaten sind die Abbildungen $t_i (x)$ wohldefiniert.
%      Definiere die Vektoren $b_i \coloneqq a_i - a_0$ und erweitere
%      das linear unabhängige System $\set{ b_i }{ 1 \leq i \leq n}$ zu
%      einer Basis des $\R^N$. Schreibe hierfür
%      $\set{ b_i }{ 0 \leq i \leq N}$. Es gilt
%      \begin{align*}
%        x - a_0 &= \sp{0}{n} t_i a_i - 1 \cdot a_0\\
%                &\overset{(*)}{=} \sp{0}{n} t_i a_i - \sp{0}{n} t_i a_0\\
%                &= \sp{1}{n} t_i \cdot (a_i - a_0)\\
%                &= \sp{1}{n} t_i b_i\\
%                &= \sp{1}{n} t_i b_i + \sp{n+1}{N} 0 \cdot b_i.
%      \end{align*}
%      Hierbei wird $(*):$ $\sp{0}{n} t_i = 1$ verwendet. Schreibe dies
%      nun als ein lineares Gleichungssystem, mit
%      $x=(x_1,\ldots,x_N),a_i=(a_i^1,\ldots,a_i^N),%
%      t=(t_1,\ldots,t_n,0,\ldots,0),B=(b_i^j)_{i,j}$.
%      Somit schreibt sich die obige Gleichung wie folgt:
%      $x-a_0 = B\cdot t$.
%     
%      Forme nach $t$ um. Dies ist möglich da $B$ als darstellende
%      Matrix von einer Basis invertierbar ist. Nun wird ersichtlich,
%      weshalb die $t_i$ für $1 \leq i \leq n$ stetige Funktionen sind.
%
%
%      Mit $t_0 = 1 - \sp{1}{n} t_i$ ist die Behauptung gezeigt.
%      % BEMERKUNG: dies zeigt auch die eindeutigkeit der t_i, da
%      % lineares gleichungssystem, eindeutige lösung
%    \end{description}
%  \end{proof}
%\end{Lem}
%
%Die Baryzentrischen als ausgezeichnete Koordinaten ermöglichen eine
%gute Art der Beschreibung der Punkte innerhalb des Simplex. Mit dieser
%Darstellung ist es leicht, möglichst viele Eigenschaften von Simplizialen
%zu zeigen.
%
%\begin{Def}[Eckmenge, Dimension, Seite, Rand, Inneres]
%  Sei $\sset{a_0,\ldots ,a_n} \subset \R^N$ ein geometrich
%  unabhängiges System und $\sigma = a_0 \ldots a_n$ ein geometrischer
%  $n$-Simplex, dann definieren wir folgende geometrische Objekte
%  \begin{enumerate}[\textbullet]
%  \item Die Menge $\{ a_0 , \ldots , a_n \}$ bezeichnet man als
%    \textbf{Eckmenge} $\V(\sigma)$ von $\sigma$.
%  \item Eine Teilmenge $\tau = b_1 \ldots b_k \subset \sigma$, mit
%    $b_i \in \V(\sigma)$, heißt \textbf{Seite}, falls $\tau$ wieder
%    einen Simplex bildet. Eine Seite heißt \textbf{echt}, falls
%    $k \neq n$.
%  % \item Eine Teilmenge $\ \subset \sigma$ heißt \textbf{Seite},
%  %   falls $\tau$ einen Simplex bildet. Eine Seite heißt \textbf{echt},
%  %   falls sie von $\sigma$ verschieden ist.
%  \item Die \textbf{Dimension} von $\sigma$ ist die Zahl $n$ bzw.
%    $\dim(\sigma) = \gr{\V(\sigma)} - 1$, wobei $\dim(\emptyset)=-1$
%    gesetzt wird.
%  \item Der \textbf{Rand} von $\sigma$ ist die folgende Menge:
%    \begin{gather*}
%      \partial\sigma \coloneqq \bigcup \; \bigl\{ \tau \; | \; \tau \text{
%          ist echte Seite von } \sigma \bigr\}.
%    \end{gather*}
%  \item Das \textbf{Innere} des Simplex ist die Menge
%    \begin{gather*}
%    	\Int(\sigma) \coloneqq \sigma \setminus \partial\sigma.
%    \end{gather*}
%  \end{enumerate}
%\end{Def}
%
%
%Um sich die vorherige Definition zu verdeutlichen wird nun ein
%anschauliches Beispiel angegeben.
%
%\begin{Bsp}
%  Wir betrachten den Simplex $\sigma^2 = e_0e_1e_2$. Mit diesem lassen
%  sich alle vorherigen Begriffsbildungen leicht anhand dieses Simplex
%  angeben.  \newline
%
%\centering
%\parbox{0.7\linewidth}{% 
%\begin{tikzpicture}
%    \node (s) at (0,1.5) {$\dim(\sigma^2)=2$};
%
%  \draw [thin,fill=lightgray] (90:1) to (210:1) to (330:1) to (90:1);
%    \node (title) at (0,0) {$\sigma^2$};
%    \fill (90:1) circle[radius=0.07cm];
%    \fill (210:1) circle[radius=0.07cm];
%    \fill (330:1) circle[radius=0.07cm];
%\end{tikzpicture}
%\hfill
%\raisebox{0.8cm}{$=$}
%\hfill
%\begin{tikzpicture}
%  \draw [thin] (90:1) to (210:1) to (330:1) to (90:1);
%    \node (title) at (0,0) {$\partial\sigma^2$};
%    \fill (90:1) circle[radius=0.07cm];
%    \fill (210:1) circle[radius=0.07cm];
%    \fill (330:1) circle[radius=0.07cm];
%
%\end{tikzpicture}
%\hfill
%\raisebox{0.8cm}{$\cup$}
%\hfill
%\begin{tikzpicture}
%  \draw [dashed,fill=lightgray] (90:1) to (210:1) to (330:1) to (90:1);
%    \node (title) at (0,0) {$\Int\sigma^2$};
%\end{tikzpicture}
%}
%\end{Bsp}
%% TODO: tikz bilder einfügen von beispielen in denen für ein beispiel symbolisch die obrigen definitionen angegeben werden
%
%Es werden nun einige Charakterisierungen und Aussagen über die
%Simplizes und deren geometrische Objekte bewiesen.
%
%\begin{Satz}\label{satz:simp}
%  \normalfont Sei $\sigma = a_0 \ldots a_n $ ein $n$-Simplex und
%  $x \in \sigma$ mit der baryzentrischen Darstellung
%  $x=\sum\limits_{i=0}^n t_i a_i$. Dann gelten folgende Aussagen
%  \begin{enumerate}[(a)]
%  \item
%    \begin{enumerate}[(i)]
%    \item
%      $x \in \partial\sigma \Leftrightarrow \exists \; 0 \leq i \leq n
%      : t_i = 0$ \label{satz:a}
%    \item
%      $x \in \Int(\sigma) \Leftrightarrow \forall \; 0 \leq i \leq n :
%      t_i > 0$
%  \end{enumerate}
%  \item Jeder Simplex $\sigma$ ist eine konvexe, kompakte Teilmenge 
%    von $\R^N$ und die konvexe Hülle von
%    $\{ a_0 \ldots a_n \}$ identisch mit dem Simplex $\sigma$.
%  \item Das Innere $\Int(\sigma)$ ist konvex und offen in $\sigma$.
%  \item Zwei Simplexe der selben Dimension sind homöomorph.
%  \item Es gibt einen Homöomorphismus $\sigma \simeq \overline{\D^n}$, der den
%    Rand $\partial\sigma$ auf die $\Sp^{n-1}$ abbildet.
%  \end{enumerate}
%  \begin{proof}
%    \begin{enumerate}[(a):]
%      % a)
%    \item Zeige nur die (i), denn die (ii) folgt unmittelbar als
%      Negation von (i) und der Tatsache, dass
%      $\sigma = \partial\sigma \cup \Int(\sigma)$ eine disjunkte
%      Vereinigung ist.
%      \begin{description}
%      \item[\glqq $\Rightarrow$\grqq] Sei $x$ im Rand
%        $\partial\sigma$. Dann liegt der Punkt in einer echten Seite
%        $\tau$ von $\sigma$. Sei \OE~ $\tau = a_0 \ldots a_m$ für ein
%        $m < n$. Dann hat $x$ eine eindeutige baryzentrische
%        Darstellung bezüglich des Simplex $\sigma$, aber auch eine
%        eindeutige Darstellung bezüglich der Seite $\tau$. Es
%        existiert also $t_i,s_i$, mit
%        \begin{gather*}
%          x = \sp{0}{m} s_i a_i = \sp{0}{n} t_i a_i .
%        \end{gather*}
%        Durch die Eindeutigkeit der Darstellung folgt unmittelbar,
%        dass ein $0 \leq i \leq n$ existieren muss, so dass $t_i = 0$
%        gilt.
%      \item[\glqq $\Leftarrow$ \grqq] Gilt nun umgekehrt $t_i = 0$ für
%        ein $0 \leq i \leq n$, so liegt der Punkt $x$ in der echten Seite
%        $a_0 \ldots a_{i-1} a_{i+1} \ldots a_n$ und damit im Rand.
%      \end{description}
%
%
%      % b)
%    \item 
%      \begin{description}
%      \item[kompakt:] Definiere eine stetige Abbildung
%        $f : \R^{n+1} \rightarrow \R^N$ mit
%        $ t_0,\ldots ,t_n \mapsto \sp{0}{n} t_i a_i$, diese ist stetig
%        als Linearkombination stetiger Funktionen.  Die Menge
%        \begin{gather*}
%          A= \bigl\{ (t_0,\ldots,t_n) \in \R^{n+1} \; | \;  \sp{0}{n} t_i = 1
%            \text{ und für alle } 0 \leq i \leq n \text{ und } t_i \geq 0 \bigr\}
%        \end{gather*}
%        ist kompakt, da man sie durch die $1$-Norm $\nn_1$ wie folgt
%        schreiben kann $A = (\nn_1)^{-1}(\{ 1 \}) \cap
%        [0,1]^{n+1}$.
%        Als Schnitt einer abgeschlossen und einer kompakten Menge ist
%        $A$ kompakt.  Somit ist $\sigma = f(A)$ als Bild einer
%        kompakten Mengen unter einer stetigen Abbildung wieder
%        kompakt.
%      \item[konvex:] Seien zwei Punkte $x,y \in \sigma$ gegeben und
%        ihre baryzentrischen Darstellungen seien $x = \sum t_i a_i$,
%        $y = \sum s_i a_i$ mit $\sum t_i = \sum s_i = 1$. Dann folgt
%        für eine Konvexkombination mit $\lambda \in [0,1]$, dass
%        \renewcommand*{\theequation}{$*$}
%        \begin{align}
%          \nonumber
%          \lambda x + (1- \lambda)y &= \lambda \cdot \sp{0}{n} t_i a_i%
%                                      + (1-\lambda) \cdot \sp{0}{n} s_i a_i \\
%                                    &= \sp{0}{n} (\lambda t_i +%
%                                      (1-\lambda) s_i) \cdot a_i.
%        \end{align}
%        Jeder Punkt dieses Verbindungsstücks von $x$ und $y$ liegt
%        wieder in $\sigma$, denn mit den Koeffizienten aus der
%        Darstellung $(*)$ folgt
%        \begin{align*}
%          \sp{0}{n} (\lambda t_i + (1-\lambda) s_i)
%          &= \lambda \cdot \sp{0}{n} t_i + (1-\lambda) \cdot \sp{0}{n} s_i \\
%          &= \lambda + (1 - \lambda) = 1.
%        \end{align*}
%        Also folgt $\lambda x + (1- \lambda)y \in \sigma$.  
%
%      \item[Konvexe Hülle:] Es ist zu zeigen, dass
%        $\conv(\sset{a_0,\ldots , a_n}) = \sigma$. Hierbei gilt
%        $\sset{a_0,\ldots,a_n} \subset \sigma$. Da $\conv$ monoton
%        ist, folgt $\conv(\sset{a_0,\ldots , a_n}) \subset \sigma$.
%        
%        Nun gilt $a_ia_j \subset \conv( \sset{a_0,\ldots , a_n} )$, da
%        mit $a_i \in \ch$ auch jeder Punkt, der sich als
%        Konvexkombinationen schreiben lässt in der konvexen Hülle
%        liegt. Rekursiv folgt nun mit
%        $\lambda \coloneqq \sp{1}{n} t_i$ und der Darstellung
%        \begin{gather*}
%          x = t_0 a_0 + \lambda \cdot \sp{1}{n} \frac{t_i}{\lambda}
%          a_0
%        \end{gather*}
%      die andere Inklusion.
%     \end{description}
%
%     % c)
%   \item \begin{description}
%     \item[offen:] Da jede Seite $\tau \subset \sigma$ wieder ein
%       Simplex ist, ist diese auch kompakt, insbesondere
%       abgeschlossen. Nun ist der Rand $\partial\sigma$ Vereinigung
%       dieser abgeschlossen Mengen, also wieder abgeschlossen. Das
%       Innere ist nun das relative Komplement der abgeschlossen Menge
%       bezüglich des Simplex, also offen in $\sigma$.
%
%     \item[konvex:] Seien zwei Punkte $x,y \in \Int(\sigma)$ gegeben,
%       dann gilt nach $b)$, dass für die baryzentrischen Koordinaten
%       der Punkte~~$t_i,s_i >0$ gilt. Sei nun $\lambda \in [0,1]$,
%       dann gilt für die baryzentrischen Koordinaten der
%       Konvexkombinationen $\lambda x + (1-\lambda y)$ mit der selben
%       Rechnung wie in $(*)$, die Darstellung
%       $\lambda t_i + (1-\lambda)s_i$. Dies ist offentsichlich stets
%       größer als $0$.
%        \end{description}
%
%        % d)
%      \item Zeige, dass jeder Simplex $\sigma$ homöomorph zum
%        Standardsimplex $e_0\ldots e_n$ ist. Wir Betrachten hierfür
%        die affine Transformation $f(x) = Ax+b$ mit $A \in \GL(n,\R)$
%        und $b \in \R^n$, die $a_i$ auf $e_i$ abbildet.  Dies ist ein
%        Homöomorphismus.
%        % Dies entspricht zunächst einer Verschiebung
%        % $x \mapsto x -a_0$. Dann wird noch eine lineare Abbildung
%        % angewendet die die Basis $\{ \}$
%
%      % \item Da $\sigma$ abgeschlossen ist und
%      %   $\overline{\hspace{0.1cm} \cdot \hspace{0.1cm}}$ ist monoton
%      %   gilt: $\overline{\Int(\sigma)} \subset \sigma$.
%
%      %   Sei nun $x \in \sigma$. Nutze die disjunkte Zerlegung des
%      %   Simplex $\sigma = \partial\sigma \cup \Int(\sigma)$. Dann
%      %   unterscheide die beiden Fälle das $x$ in genau einem der
%      %   beiden Mengen liegt. Für
%      %   $x \in \Int(\sigma) \subset \overline{\Int(\sigma)}$ sind wir
%      %   schon fertig. Für $x \in \partial\sigma$ nutze die
%      %   Charakterisierung der Elemente des Randes, also
%      %   $x \in \partial A \Leftrightarrow \forall U \text{ offene
%      %     Umgebung von } x : U \cap A \not= \emptyset \text{ und } U
%      %   \cap A^c \not= \emptyset $.
%      %   Und mithilfe der Charakterisierung der Elemente des
%      %   Abschlusses:
%      %   $ x \in \overline{A} \Leftrightarrow \forall U \text{ offene
%      %     Umgebung von }x : U \cap A \not= \emptyset$.
%      %   Und somit folgt die Behauptung.
%
%      %   e)
%      \item Wir betrachten die stetige Abbildung
%        $f : \R^{n+1} \setminus \sset{0} \rightarrow \Sp^{n}, f(x) =
%        \frac{x}{\nn[x]}$.
%        Sei mit $a \in \Int(\sigma), p \in \R^N \setminus \{ 0 \}$ die
%        Halbgerade
%        $ap_+ \coloneqq \bigl\{ a + tp \; | \; t \geq 0 \bigr\}$
%        gegeben. Dann beweise folgende Behauptung.
%      \begin{Beh}
%        Der Schnitt $\partial\sigma \cap ap_+$ hat genau ein Element.
%        \begin{proof}
%          Zur Existenz des Elements betrachte die Menge
%          $ap_+ \cap \Int(\sigma)$.  Diese ist homöomorph zu $[0,b)$
%          für ein $b>0$ und somit existiert bezüglich des
%          Homöomorphismus durch den Punkt $b$ ein Schnittpunkt der
%          Halbgerade mit dem Rand.
%          
%          Seien nun zwei Punkte $x,y \in \partial\sigma$ gegeben, die
%          auch in $ap_+$ enthalten sind. Seien $t_x, t_y \in \R_{> 0}$
%          die Parameter, sodass $a + t_z \cdot p = z$ mit
%          $z = x,y$ gilt. Sei \OE~ $t_y < t_x$, dann folgt
%          durch eleminieren von $p$, dass
%          \begin{gather*}
%            x = (1-t) a + ty \text{ mit } t=\frac{t_y}{t_x} < 1.
%          \end{gather*}
%          Wir wählen nun eine Folge $y_n \in \Int(\sigma)$ die gegen $y$
%          konvergiert und definiere
%          $a_n \coloneqq \frac{1}{1-t} (x -ty_n)$.  Dann konvergiert
%          nach Konstruktion $a_n$ gegen $a$ und da
%          $a \in \Int(\sigma)$ in einer offenen Menge enthalten ist,
%          existiert ein $n \geq 0$, sodass für alle $m \geq n$ gilt
%          $a_m \in \Int(\sigma)$.  Somit gilt
%          $x = (1-t) a_n + ty_n \in \Int(\sigma)$, da die Menge konvex
%          ist und die Folgenglieder ab einem $m$ in der Menge
%          $\Int(\sigma)$ liegen, folgt ein Widerspruch zu
%          $x \in \partial\sigma$.
%        \end{proof}
%      \end{Beh}
%      Sei nun \OE~ $0 \in \Int(\sigma)$, sonst wende eine affine
%      Transformation auf $\sigma$ an, die den Nullpunkt in das Innere
%      von $\sigma$ verschiebt. Dies ändert nichts an der folgenden
%      Argumentation.
%
%      Die Einschränkung der Abbildung $f$ auf die Menge
%      $\partial\sigma$ ein. Nach obiger Behauptung ist die Abbildung
%      $f_{| \partial\sigma} : \partial\sigma \rightarrow \Sp^{n-1}$
%      bijektiv und stetig. Da $\partial\sigma$ und $\Sp^{n-1}$ kompakt
%      und hausdorffsch sind, ist die Abbildung auch ein
%      Homöomorphismus.
%      
%      Setze nun als Umkehrabbildung $g : \Sp^{n-1} \rightarrow \partial\sigma$ 
%      und erweitere diese zu einem Homöomorphismus der Form
%      \begin{gather*}
%        G : \overline{\D^n} \rightarrow \sigma , \hspace{0.5cm}
%        x \mapsto 
%        \begin{cases}
%          \hspace{.7cm}0, &  x = 0. \\
%          ~\bigl\| g(\frac{x}{\nn[x]}) \bigr\|
%%\left\| #1 \right\|
%          \cdot x, & \text{ sonst}.
%        \end{cases}
%      \end{gather*}
%      Diese Abbildung erfüllt die gewünschten Eigenschaften.
%    \end{enumerate}
%  \end{proof}
%\end{Satz}
%
%
%\begin{Bem}
%  Für eine beliebige Indexmenge $J$ ist der Raum der Funktionen $\R^J$ ein
%  $\R$-Vektorraum, dessen Elemente als Tupel $(x_j)_{j \in J}$
%  geschrieben werden. Betrachte den Untervektorraum
%  $\E^J \coloneqq \bigoplus\limits_{j \in J} \R$ der Elemente, die bis
%  auf endlich viele von Null verschieden sind. Definiere für zwei
%  Elemente $x,y \in \E^J$ eine Metrik
%  \begin{gather*}
%    \gr{x-y} \coloneqq \sup \set{ \gr{x_j - y_j } }{ j \in J}.
%  \end{gather*}
%  Die obigen Definitionen und Aussagen funktionieren ebenso falls man
%  die Simpliziale in dem Raum $\E^J$ betrachtet. Somit ist sind auch
%  geometrische Simpliziale der Kardinalität größer als der von $\R$
%  möglich.
%\end{Bem}
%
%
%
%
%%%% Local Variables:
%%%% mode: latex
%%%% TeX-master: "main"
%%%% End:
