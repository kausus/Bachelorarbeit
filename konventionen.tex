%!TEX root = main.tex
% vim: tw=0 noet sts=8 sw=8



%kapitel Einführung und hinführung zum thema, erklärung von konventionen
\section*{Einführung}

In dieser Bachelorarbeit soll ein nichttrivialer Zusammenhang zwischen
dem Raum aller Immersionen einer zweidimensionalen, kompakten, orientierten Mannigfaltigkeit und der Menge aller Spinstrukturen (bis auf Isomorphie) auf dieser Mannigfaltigkeit
bewiesen werden.

Im ersten Teil der Arbeit werden die notwendigen Begriffe wie Spinstruktur
und Immersionen eingeführt und anhand von Beispielen vertieft.

\section{Konventionen}
Es werden in dieser Arbeit die Begriffe von Mannigfaltigkeit, Liegruppe,
riemanischer Metrik 
% TODO: was noch, hier werden begriffe gesammelt die nicht aufgrund der Länge definiert werden sollen, aber trotzdem verwendet werden

Eine Mannigfaltigkeit meint hier eine Menge mit einem glatten Atlas, sodass die von der glatten Struktur erzeuge Topologie das zweite Abzählbarkeitskriterium erfüllt und parakompakt ist.

%In diesem Seminarvortrag werden folgende Konventionen verwendet. Die
%Null ist eine natürliche Zahl. Die reellen Zahlen werden durch $\R$
%geschreiben.  Eine Menge $A \subset \R^n$ ist konvex, wenn die
%Verbindungslinie je zweier Punkte vollkommen in $A$ enthalten ist.
%
%Die konvexe Hülle einer Teilmenge $A \subset \R^n$ eines
%$\R$-Vektorraums, ist der Schnitt aller konvexen Mengen, die diese
%Menge enthalten. Diese ist konvex. Schreibe hierfür $\conv(A)$.
%
%Es bezeichnet $e_i$ den $i$-ten Einheitsvektor in $\R^n$, wobei $e_0$
%als der Nullvektor gesetzt wird.
%
%Eine Konvexkombination zweier Vektoren $x$ und $y$, mit Parametern
%$\lambda,\mu \in [0,1]$, ist eine Linearekombination
%$\lambda x + \mu y$ mit $\lambda + \mu = 1$.
%
%Die Potenzmenge wird durch $\pow$ geschrieben.
%
%Für eine Menge $A \subset X$ aus einem topologischen Raum, bezeichnet
%$\overline{A}$ den Abschluss von $A$.
%
%Das Zeichen \OE~ bedeutet ohne Einschränkung.
%
%Die $1$-Norm $\nn_1$ auf dem $\R^N$ ist durch folgende Abbildung gegeben:
%\begin{gather*}
%  \nn_1 : \R^N \rightarrow \R_{\geq 0}\\
%  x = (x_1,\ldots ,x_N) \mapsto \sp{1}{N} |x_i|
%\end{gather*}


%%% Local Variables:
%%% mode: latex
%%% TeX-master: "main"
%%% End:
