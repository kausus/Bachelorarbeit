%!TEX root = main.tex
% vim: tw=0 noet sts=8 sw=8


%erstes Kapitel
\section{Hauptfaserbündel und Spinstruktur}

In diesem ersten Abschnitt soll die grundlegenden Strukturen eingeführt werden die für diese Arbeit notwendig sind. Der Begriff des Hauptfaserbündel hat sich als der Richtige herausgebildet indem die Sprache von Spinstrukturen bzw. Spinoren formuliert wird.

Die verwendete Quelle ist \cite{baum09}

\begin{Def}[Hauptfaserbündel]
	Ein \textit{Hauptfaserbündel} ist ein Tupel $(P,\pi,M,G)$ das aus folgenden Daten besteht
	\begin{enumerate}[\textbullet]
		\item Die Mengen $P,M$ sind \mfgen, die Abbildung $ \abb{\pi}{P}{M} $ ist eine lokale Trivialiserung und $G$ ist eine Liegruppe
		die auf $P$ von rechts, fasertreu, transitiv und frei wirkt. 
		\nomenclature{frei}{eine Gruppenwirkung ist frei, falls die Rechtsmultiplikation mit dem Element $g$ fixpunktfrei ist falls $g$ schon trivial ist}
		\nomenclature{fasertreu}{Die Faser ist unter Gruppenwirkung stabil, also werden Faserelemente auf Faserelemente abgebildet}
		\item Es gibt einen Bündelatlas $\sset{(U_i,\phi_i)}$ aus $G$-äquivarianten Bündelkarten, d.h.
		Es gilt das $\sset{(U_i)_{i \in \Lambda}}$ eine offene
		Überdeckung von M ist. Die Abbildung $ \abb{\phi_i}{\pi^{-1}(U_i)}{U_i\times G} $ ist ein
		Diffeomorphismus für alle $i \in \Lambda $, für den $ \phi_i(pg) = \phi(p) g  $ und $ \pi_1 \circ \phi_i = \pi^P $ für alle $ g \in G, p \in \pi^{-1}(U_i) $ gilt. Wobei $ (p,g) h = (p,gh) $ in 
		$ U_i\times G $ gilt. 
	\end{enumerate}		
	Für zwei Hauptfaserbündel $(P,\pi^P,M,G),(Q,\pi^Q,M,H)$ 
	über derselben Mannigfaltigkeit und einer Abbildung $\abb{f}{P}{Q}$ nennen wir $f$ einen \textit{Hauptfaserbündelmorphismus}, falls $ \pi^P=\pi^Q \circ f $ gilt. Ein Hauptfaserbündelisomorphismus ist
	ein Isomorphismus im Sinne der Kategorientheorie.
\end{Def}

Im folgenden werden die wichtigsten Beispiele angegeben.

\begin{Bsp}
	\begin{enumerate}[1)]
		\item Sei $G$ eine Liegruppe und $M$ eine Mannigfaltigkeit, dann heißt das Tupel $(M \times G, \pi, M, G)$ das triviale Hauptfaserbündel zu $M$ und $G$, wobei die Abbildung
		durch die Projektion $\abb{\pi}{M \times G}{M}$ , $(p,g) \mapsto p$ 
		gegeben ist. 
		\begin{smallproof}(Nachweis das dies ein Hauptfaserbündel ist)
		Die Gruppenwirkung ist durch folgende Abbildung gegeben:
		\DefMap{M\times G}{M}{((p,g),h)}{(p,gh)}
		
		Diese Gruppenwirkung ist die der Gruppe auf sich selbst
		und erfüllt damit offentsichlich die Eigenschaften
		transitiv und frei. Die Faser eines Elements $p\in M$ 
		ist $\{p\}\times G$ und damit gilt offentsichlich
		durch die Definition der Gruppenwirkung die Fasertreue.
		
		Die Abbildung ist zudem eine lokale Trivalisierung mit
		$G$-äquivaranten Bündelkarten. Sei hierzu $\mathcal{A}=\sset{(U_i,\phi_i)_{i \in \Lambda}}$ ein
		Atlas von $M$ gegeben, dann ist $ \sset{\abb{\id\times\id}{U_i \times G}{U_i\times G}}{i \in \Lambda} $ ein Bündelatlas	von $G$-äquivarianten Bündelkarten.
		Damit ist dies ein Hauptfaserbündel mit Strukturgruppe $G$.
		\end{smallproof}
		
		\item Das wohl wichtigste Beispiel für ein Hauptfaserbündel in dieser
		Arbeit ist das \textit{Rahmen- oder Reperebündel}. Sei $M$ eine \mfg und $p \in M$,dann definere $\GL_p(M)$ als die Menge aller Basen von
		$\tang{p}{M}$ und $\GL(M) \coloneqq \amalg_{p \in M} \GL_p(M)$. 
		Betrachte nun das Tupel $(\GL(M),\pi,M,\GL_n)$, wobei die Abbildung
		durch $\abb{\pi}{\GL(M)}{M}$ , $((v_i),p) \mapsto p$ gegeben ist
		und die Liegruppe $\GL_n$ von rechts auf die \mfg $\GL(M)$ wie  folgt
		wirkt:
		\begin{center}
			\begin{tikzcd}
				\GL(M) \times \GL_n \arrow[r]  & \GL(M) \\
				(((v_1,\ldots,v_m),p),A) \arrow[r,mapsto] & ((v_1 A,\ldots,v_n A),p)
			\end{tikzcd}
		\end{center}
		\begin{smallproof}
			Zunächst ist ein Atlas für die Menge $ \GL(M) $ anzugeben. Sei hierzu $\mathcal{A}=\sset{(U_i,\phi_i)_{i\in \Lambda}}$
			ein Atlas von $M$. 
			\todo{Angabe eines Atlas für $\GL(M)$}
			
			Die Gruppenwirkung ist transitiv, da die Elemente
			von $\GL(M)$ gerade Basen eines Vektorraums sind
			und sich durch die entsprechenden Basistransformationsmatrizen ineinander überführen
			lassen. Die Eigenschaft frei wollt auch aus
			der Eigenschaft von Basistransformationen. Die
			Fasertreue ist auch anhand der Definition klar, 
			da die Gruppenwirkung innerhalb des Tangentialraums
			nur transformiert.
			
			Sei nun ein Bündelatlas $ \sset{(U_i,\phi_i)_{i\in \Lambda}} $ von $ \T M $ gegeben. Betrachte
			nun die Abbildungen
			\DefMap{\pi^{-1}(U_i)}{U_i\times \GL_n}{((v_i,p))}{(p,\phi_i(v_i))}. Diese liefern einen
			$ \GL_n $-äquivarianten Bündelatlas für das Rahmenbündel.
			
		\end{smallproof}
		
	\end{enumerate}
\end{Bsp}


Im folgenden wird der Begriff der Spinstruktur benötigt, es wird
hierzu lediglich die Defintion und weitere Beispiele angegeben.
Die verwendete Quelle ist \cite{BHMMM15}.
\todo{sollen äquivalente Definitionen von Spinstruktur genannt werden? zb. cozykel-liften, oder topologische Spinstrukturen}
\todo{ist es klar dass das paar $(\phi...)$ eine mfg ist? soll das diagramm größer gemacht werden?}
\begin{Def}[Spinstruktur]
	Sei $(M,g)$ eine \RMF der Dimension $m$, dann ist
	 ($\Ph{\spin{n}}{M}$,$\eta : \Ph{\spin{n}}{M} \rightarrow \Ph{\so{n}}{M}$) eine Spinstruktur auf $(M,g)$, falls folgndes gilt:
	 \begin{itemize}
	 	\item Die \mfg $\Ph{\spin{n}}{M}$ ist ein $\spin{n}$-Hauptfaserbündel.
	 	\item Die Abbildung $\eta$ ist eine doppelte Überlagerung.
	 	\item Folgendes Diagramm kommutiert\footnote{Dies ist ein Spezialfall für eine Reduktion eines Hauptfaserbündel.}\\
		 	\begin{center}
		 		% das zeichen ' nach " " ändert die position bzgl pfeil
		 	\begin{tikzcd}
		 		\spin{n} \ar[rr, "a \mapsto ua"] \ar[dd, "\Ad"'] & &  \Ph{\spin{n}}{M}  \arrow[rd] \arrow[dd, "\eta"] & \\
		 		&&& M \\
		 		\so{n} \arrow[rr , "A \mapsto \eta(u)A"] & & 
		 		\Ph{\so{n}}{M} \arrow[ur] &
		 	\end{tikzcd}
			 \end{center}
			wobei $\Ad$ die doppelte Überlagerungsabbildung von $\so{n}$ ist
			und dies für alle $u \in \Ph{\spin{n}}{M}$ gilt.
	 \end{itemize}
	 Falls eine \RMF eine Spinstruktur besitzt sagen wir die \mfg ist \textit{spin}.
	

\end{Def}

\begin{Bsp}
	\begin{enumerate}[(1)]
		\item Das erste Beispiel ist das Triviale. Für die \RMF $(\R^n, \scdots_{\R^n})$ ist $(\R^n \times \spin{n},\id \times \Ad)$ 
		eine Spinstruktur auf dieser riemannischen 
		Mannigfaltigkeit.\footnote{Dies ist auch die einzigste Spinstruktur für den $ \R^n $ bis auf Isomorphie}. Allgemein gilt das falls 
		das Orientierungsbündel \todo{falscher Name} $ \Ph{\so{n}}{M} $ für eine
		Mannigfaltigkeit $ M^n $ trivial ist, also isomorph zu $ M\times \spin{n} $ ist, dann ist das triviale $ \spin{n} $-Haupfaserbündel
		$ (M\times \spin{n},\pi_1,M,\spin{n}) $ eine Spinstruktur für $ M $.
		Doch wie wir im nächsten Beispiel sehen werden folgt daraus noch nicht
		das dies dann die einzige Spinstruktur sein muss. (bis auf Isomorphie)
		\begin{proof}
			Das Tupel $ (\R^n\times \spin{n},\pi_1,\R^n,\spin{n}) $ ist ein triviales
			Hauptfaserbündel mit Strukturgruppe $ \spin{n} $.
			Sei ein $ (p,u)\in \R^n\times \spin{n} $ gegeben, brachte dann
			folgendes Diagramm
			
%			\begin{center}
%				\begin{tikzcd}
%				\spin{n} \arrow[rr, "a \mapsto (p,au)"] \arrow[dd,"\Ad"] && \R^n\times \spin{n} \arrow[dd] \aarow  & \\
%				&&& M \\
%				\so{n} && \R^n &
%			\end{tikzcd}
%			\end{center}	
				\begin{center}
					
					 	\begin{tikzcd}
					 		\spin{n} \ar[rr, "a \mapsto \pair{p}{au}"] \ar[dd, "\Ad"'] & &  \R^n\times \spin{n}  \arrow[rd] \arrow[dd, "\pi_1"] & \\
					 		&&& \R^n \\
					 		\so{n} \arrow[rr , "A \mapsto pA"] & & 
					 		\R^n \arrow[ru] &
					 	\end{tikzcd}
		\end{center}

		\end{proof}
		
		\item Ein erstes nichttriviales Beispiel sind zwei nicht
		isomorphie Spinstrukturen der $\S^1$. Diese sind $(\S^1 , \eta_0)$ und $(\S^1 \times \Z_2 = \S^1 \times \spin{1j} , \eta_1)$, wobei die Abbildungen durch $\eta_0 : \S^1 \rightarrow \S^1 , z \mapsto z^2$ und $\eta_1 : \S^1 \times \Z_2 \rightarrow \S^1 , (z,\alpha) \mapsto z$ gegeben sind. Diese sind auch die Repräsentaten aller Spinstrukturen bis auf 
		Isomorphie. 
		\todo{suche verweis auf hijazi}
		\item Die $(\CP^{2n},g)$ ist ein Beispiel für eine Familie von
		\RMFen die nicht spin sind, wobei
		hier $g$ eine riemannische Metrik
		auf $\CP^{2n}$ ist. 
\todo{Quelle einfügen}
		\item Ein wichtiges Beispiel für eine \RMF mit deren Spinstruktur ist die $\S^2$. Diese
		besitzt genau eine Spinstruktur bis auf Isomorphie. 
\todo{angabe dieser spinstruktur}
\todo{Angabe von Spinstrukturen auf S1 und Torus,alle vier Spinstrukturen?}
	\end{enumerate}
\end{Bsp}






%% Local Variables:
%% mode: latex
%% TeX-master: "main"
%% End:
