%!TEX root = main.tex
% vim: tw=0 noet sts=8 sw=8


%erstes Kapitel
\section{Hauptfaserbündel und Spinstruktur}
\label{sec:Hauptfaserbündel und Spinstruktur}

In diesem ersten Abschnitt soll die grundlegenden Strukturen
eingeführt werden die für diese Arbeit notwendig sind. Der Begriff des
Hauptfaserbündel hat sich als der Richtige herausgebildet indem die
Sprache von Spinstrukturen bzw. Spinoren formuliert wird.

Die verwendete Quelle ist \cite{baum09}.

\begin{Def}[Hauptfaserbündel]
  Ein \textit{Hauptfaserbündel} ist ein Tupel $(P,\pi,M,G)$ das aus
  folgenden Daten besteht
	\begin{enumerate}[\textbullet]
        \item Die Mengen $P,M$ sind \mfgen, die glatte Abbildung
          $ \abb{\pi}{P}{M} $ ist eine lokale triviale Abbildung und
          $G$ ist eine Liegruppe die auf $P$ von rechts, fasertreu,
          transitiv und frei wirkt.  \nomenclature{frei}{eine
            Gruppenwirkung ist frei, falls die Rechtsmultiplikation
            mit dem Element $g$ fixpunktfrei ist falls $g$ schon
            trivial ist} \nomenclature{fasertreu}{Die Faser ist unter
            Gruppenwirkung stabil, also werden Faserelemente auf
            Faserelemente abgebildet}
        \item Es gibt einen Bündelatlas $\sset{(U_i,\phi_i)}$ aus
          $G$-äquivarianten Bündelkarten, d.h.  Es gibt
           eine offene Überdeckung $\sset{(U_i)_{i \in \Lambda}}$ von $ M $ und  Abbildungen
          $ \abb{\phi_i}{\pi^{-1}(U_i)}{U_i\times G} $ die
          Diffeomorphismen sind für alle $i \in \Lambda $, für die
          $ \phi_i(pg) = \phi(p) g $ und
          $ \pi_1 \circ \phi_i = \pi^P $ für alle
          $ g \in G, p \in \pi^{-1}(U_i) $ gilt. Wobei
          $ (p,g) h = (p,gh) $ in $ U_i\times G $ gilt.
	\end{enumerate}		
	Für zwei Hauptfaserbündel $(P,\pi^P,M,G),(Q,\pi^Q,M,H)$ über
        derselben Mannigfaltigkeit und einer Abbildung $\abb{f}{P}{Q}$
        nennen wir $f$ einen \textit{Hauptfaserbündelmorphismus},
        falls $ \pi^P=\pi^Q \circ f $ und mit $ G $-äquivariant ist, gilt. Ein Hauptfaserbündelisomorphismus ist ein Isomorphismus im Sinne
        der Kategorientheorie.
\end{Def}

Im folgenden werden die wichtigsten Beispiele angegeben.

\begin{Bsp}
  \begin{enumerate}[1)]
  \item Sei $G$ eine Liegruppe und $M$ eine Mannigfaltigkeit, dann
    heißt das Tupel $(M \times G, \pi, M, G)$ das triviale
    Hauptfaserbündel zu $M$ und $G$, wobei die Abbildung durch die
    Projektion $\abb{\pi}{M \times G}{M}$ , $(p,g) \mapsto p$ gegeben
    ist.
    \begin{smallproof}(Nachweis das dies ein Hauptfaserbündel ist) Die
      Gruppenwirkung ist durch folgende Abbildung gegeben:
      \DefMap{M\times G}{M}{((p,g),h)}{(p,gh)}
		
      Diese Gruppenwirkung ist die der Gruppe auf sich selbst und
      erfüllt damit offentsichlich die Eigenschaften transitiv und
      frei. Die Faser eines Elements $p\in M$ ist $\{p\}\times G$ und
      damit gilt offentsichlich durch die Definition der
      Gruppenwirkung die Fasertreue.
		
      Die Abbildung ist zudem eine lokale Trivalisierung mit
      $G$-äquivaranten Bündelkarten. Sei hierzu
      $\mathcal{A}=\sset{(U_i,\phi_i)_{i \in \Lambda}}$ ein Atlas von
      $M$ gegeben, dann ist
      $ \sset{\abb{\id\times\id}{U_i \times G}{U_i\times G}}_{i \in
        \Lambda} $
      ein Bündelatlas von $G$-äquivarianten Bündelkarten.  Damit ist
      dies ein Hauptfaserbündel mit Strukturgruppe $G$.
		\end{smallproof}
		
              \item Das wohl wichtigste Beispiel für ein
                Hauptfaserbündel in dieser Arbeit ist das
                \textit{Rahmen- oder Reperebündel}. Sei $M$ eine \mfg
                und $p \in M$,dann definere $\GL_p(M)$ als die Menge
                aller Basen von $\tang{p}{M}$ und
                $\GL(M) \coloneqq \amalg_{p \in M} \GL_p(M)$.
                Betrachte nun das Tupel $(\GL(M),\pi,M,\GL_n)$, wobei
                die Abbildung durch $\abb{\pi}{\GL(M)}{M}$ ,
                $((v_i),p) \mapsto p$ gegeben ist und die Liegruppe
                $\GL_n$ von rechts auf die \mfg $\GL(M)$ wie folgt
                wirkt:
		\begin{center}
			\begin{tikzcd}
				\GL(M) \times \GL_n \arrow[r]  & \GL(M) \\
				(((v_1,\ldots,v_m),p),A) \arrow[r,mapsto] & ((v_1 A,\ldots,v_n A),p)
			\end{tikzcd}
		\end{center}
		\begin{smallproof}
			
                  Zunächst ist ein Atlas für die Menge $ \GL(M) $
                  anzugeben. Sei hierzu
                  $\mathcal{A}=\sset{(U_i,\phi_i)_{i\in \Lambda}}$ ein
                  Atlas von $M$.  
                  
                  Die Gruppenwirkung ist transitiv, da die Elemente
                  von $\GL(M)$ gerade Basen eines Vektorraums sind und
                  sich durch die entsprechenden
                  Basistransformationsmatrizen ineinander überführen
                  lassen. Die Eigenschaft frei folgt unmittelbar auch aus den Eigenschaften von Basistransformationen. Die
                  Fasertreue ist auch anhand der Definition klar, da
                  die Gruppenwirkung innerhalb des Tangentialraums nur
                  transformiert.
			
                  Sei nun ein Bündelatlas \todo{was ist ein bündelatlas}
                  $ \sset{(U_i,\phi_i)_{i\in \Lambda}} $ von $ \T M $
                  gegeben. Betrachte nun die Abbildungen
                  \DefMap{\pi^{-1}(U_i)}{U_i\times
                    \GL_n}{((v_i,p))}{(p,\phi_i(v_i)).} Diese liefern
                  einen $ \GL_n $-äquivarianten Bündelatlas für das
                  Rahmenbündel.
			
		\end{smallproof}
              \item Analog zum vorherigen Beispiel existiert zu einer
                fest gewählten Orientierung das
                \textit{orientierte Reperebündel}
                $ (\so(M),\pi,M,\so_n) $, definiert durch
		\begin{gather*}
			\so(M) \coloneqq \bigcup_{p\in M} \set{(v_1,\ldots,v_n)}{\text{positiv orientierte ONB von }\tang{p}{M}}.
		\end{gather*}
		Wir schreiben auch $ \Ph{\so_n}{M} $ anstatt $ \so(M)
                $.
	\end{enumerate}
\end{Bsp}


Im folgenden wird der Begriff der Spinstruktur benötigt, es wird
hierzu lediglich die Defintion und weitere Beispiele angegeben.  Die
verwendete Quelle ist \cite{BHMMM15}.  \todo{sollen äquivalente
  Definitionen von Spinstruktur genannt werden? zb. cozykel-liften,
  oder topologische Spinstrukturen} \todo{ist es klar dass das paar
  $(\phi...)$ eine mfg ist? soll das diagramm größer gemacht werden?}
\begin{Def}[Spinstruktur]\label{DefSpin}
  Sei $(M^n,g)$ eine orientierte, riemannsche Mannigfaltigkeit, dann ist ein Paar
  ($\Ph{\spin_n}{M}$,$\eta : \Ph{\spin_n}{M} \rightarrow
  \Ph{\so_n}{M}$) eine Spinstruktur auf $(M,g)$, falls folgendes gilt:
	 \begin{itemize}
         \item Die \mfg $\Ph{\spin_n}{M}$ ist ein
           $\spin_n$-Hauptfaserbündel über $ M $.
         \item Folgendes Diagramm kommutiert\footnote{Dies ist ein Spezialfall für eine Reduktion eines Hauptfaserbündel siehe \cite{baum09}.}\\
%           \begin{center}
%             % das zeichen ' nach " " ändert die position bzgl pfeil
%		 	\begin{tikzcd}
%                          \spin_n \ar[rr, "a \mapsto ua"] \ar[dd, "\Ad"'] & &  \Ph{\spin_n}{M}  \arrow[rd] \arrow[dd, "\eta"] & \\
%                          &&& M \\
%                          \so_n \arrow[rr , "A \mapsto \eta(u)A"] & &
%                          \Ph{\so_n}{M} \arrow[ur] &
%		 	\end{tikzcd}
%			 \end{center}

           \begin{center}
           	% das zeichen ' nach " " ändert die position bzgl pfeil
           	\begin{tikzcd}
           		\spin_n \arrow[d,"\Ad"] \arrow[out=15,in=-15,loop] & \Ph{\spin_n}{M} \arrow[dr] \arrow[d,"\eta"] & \\
           		\so_n \arrow[out=15,in=-15,loop] & \Ph{\so_n}{M} \arrow[r] & M
           	\end{tikzcd}
           \end{center}

                         wobei $\Ad$ die doppelte
                         Überlagerungsabbildung von $\so_n$ ist und
                        die gekrümmten Pfeile die glatten Gruppenwirkungen sind.
                         Es folgt nun schon das die Abbildung $\eta$   eine doppelte Überlagerung ist.
	 \end{itemize}
	 Falls eine \RMF eine Spinstruktur besitzt sagen wir die \mfg
         ist \textit{spin}.
     Zwei Spinstrukturen $ (\Ph{\spin_n}{M},\eta),(P_{\spin_n}^{'}M,\eta^{'}) $ über $ M $ heißen isomorph, falls einen $ \spin_n $-Hauptfaserbündelisomorphismus $ \abb{\beta}{\Ph{\spin_n}{M}}{P_{\spin_n}^{'}M} $ gibt, sodass die
     Spinstrukturen miteinander verträglich sind, also $ \eta=\eta^{'} \circ \beta $ gilt\footnote{Es existieren Hauptfaserbündel die als lokal-trivale Faserungen isomorph sind, aber nicht als Hauptfaserbündel}.
\end{Def}

Die Frage nach der Existenz und Eindeutigkeit von Spinstrukturen 
auf einer riemannschen Mannigfaltigkeit $ (M,g) $ wird durch folgenden
Satz beantwortet.

\begin{Satz}\label{anzahlspin}
	Sei $ (M,g) $ eine orientierte, riemannsche Mannigfaltigkeit dann
	gilt das $ H^1(M,\Z_2) $ isomorph ist zu der Menge aller $ \Z_2 $-Hauptfaserbündel über $ M $. Zudem operiert $ H^1(M,\Z_2) $ frei
	und transitiv auf der Menge aller Spinstrukturen. 
	\begin{proof}
%		$ \tilde{M} \times_{\pi_1(M)} \Z_2 $
%		oder
%		$ \tilde{M} \times \Z_2 / \pi_1(M)$
		Wir können $ H^1(M,\Z_2)\simeq \Hom(\pi_1(M),\Z_2) $ aufgrund des universellen
		Koeffiziententheorems identifizieren. Wir nutzen nun das $ \pi_1(M) $
		auf der universellen Überlagerung von $ M $ operiert.
		
		Sei nun $ \chi\in\Hom(\pi(M),\Z_2) $ gegeben und wir wollen
		ein $ \Z_2 $-Hauptfaserbündel konstruieren. Sei $ \abb{\pi}{\tilde{M}}{M} $ die universelle Überlagerung
		und $ \tilde{M}\times_{\pi_1(M)}\Z_2 $ ist gegeben als
		Quotient bezüglich der Äquivalenzrelation : $ (p,a) \sim (p\gamma,\chi(\gamma)^{-1}a) $ für alle $ p\in\tilde{M},a\in\Z_2,\gamma\in\pi_1(M) $.
		Nun ist ein $ \Z_2 $-Hauptfaserbündel gegeben durch
		 \DefMap{\pi_\chi \colon \tilde{M} \times_{\pi_1(M)} \Z_2}{M}{\left[p,a\right]}{\pi(p)}
		Die Wohldefiniertheit der Projektion folgt aus der Faserinvarianz
		von $ \pi $.
		
		Seien nun zwei $ \chi_1,\chi_2\in H^1(M,\Z_2) $ gegeben mit $ \pi_{\chi_1}=\pi_{\chi_2} $. Dann folgt unmittelbar mit $ [p\gamma,\chi_1(\gamma)a] = [p,a] = [p\gamma,\chi_2(\gamma)a] $
		die Gleichheit $ \chi_1=\chi_2 $.
		
		Sei nun ein $ \Z_2 $-Hauptfaserbündel $ \Ph{\Z_2}{M} $ über $ M $
		gegeben, dann ist ein Gruppenhomomorphismus $ \chi\in\Hom(\pi_1(M),\Z_2) $ mit $ \tilde{M}\times_{\pi_1(M)} \Z_2 = \Ph{\Z_2}{M}$ anzugeben.\todo{surjektivität}
		
		Wir wollen nun die Gruppenwirkung von $ H^1(M,\Z_2)\simeq \Ph{\Z_2}{M} $ auf der Menge aller Spinstrukturen bis auf Isomorphie definieren,
		diese ist gegeben durch
		\DefMap{H^1(M,\Z_2) \times \sset{\text{Spinstrukturen}}/\sim}{\sset{\text{Spinstrukturen}}/\sim}{(\Ph{\spin_2}{M},\Ph{\Z_2}{M})}{\Ph{\spin_2}{M}\times_{\Z_2} \Ph{\Z_2}{M}. }
		Wobei wir $ \Z_2=\spin_1\subset\spin_2 $ und damit eine $ \spin_2 $-
		Gruppenwirkung auf einem $ \spin_2 $-Hauptfaserbündel genutzt haben.
		
		Wir zeigen nun das dies eine freie, transitive Gruppenwirkung ist. Sei hierzu $ \abb{(\pi\circ p_1)}{\tilde{M}\times \Z_2}{M} $ das triviale $ \Z_2 $-Hauptfaserbündel. Mit der Definition gilt nun
		$ \Ph{\spin_2}{M} \times_{\Z_2} (\tilde{M} \times \Z_2) = \Ph{\spin_2}{M} $.\todo{genauer}
		
		Für zwei $ \Z_2 $-Hauptfaserbündel $ \Ph{\Z_2}{M},P^{'}_{\Z_2}M $
		gilt nun $ ( \Ph{\spin_2}{M} \times_{\Z_2} P^{'}_{\Z_2}M ) \times_{\Z_2} \Ph{\Z_2}{M} = \Ph{\spin_2}{M} \times_{\Z_2} (P^{'}_{\Z_2}M \times_{\Z_2} \Ph{\Z_2}{M}) $.
		
		\todo{frei und transitiv}
		
	\end{proof}
\end{Satz}

Mithilfe dieses Satzes folgt unmittelbar das
\begin{Kor}
	Für eine $ 1 $-zusammenhängende\footnote{Eine Mannigfaltigkeit $ M $ ist $ n $-zusammenhängend falls $ \pi_k(M)=0 $ für $ 0\leq k\leq n $.}, spin\footnote{Man muss die Existenz von Spinstrukturen hier fordern} Mannigfaltigkeit existiert genau eine Spinstruktur bis auf Isomorphie. Für eine Fläche mit Geschlecht $ \gamma $ existieren
	genau $ 2^{2\gamma} $ nicht isomorphe Spinstrukturen. Allgemein
	gilt für eine spin Mannigfaltigkeit das diese $ \# H^1(M,\Z_2) $-Spinstrukturen bis auf Isomorphie hat.
	\begin{proof}
		Wir zeigen die letzte Aussage, da die vorherigen Spezialfälle
		davon sind. Wir müssen lediglich eine simple Tatsache über
		freie, transitive Gruppenwirkungen auf einer nichtleeren Menge wissen. Sei $ G\times M \longrightarrow M $ die Gruppenwirkung,
		dann gilt das man zu jedem Element $ m\in M $ sich alle
		anderen Elemente von $ M $ durch die Gruppenwirkung erreichen lassen. Somit gilt $ M = \set{gm}{g\in G} $. Die Freiheit
		der Gruppenwirkung liefert uns nun aus $ gm=hm $ schon die 
		Gleichheit $ g=h $. Somit sind alle Element in der Darstellung
		$ \set{gm}{g\in G} $ verschieden und damit gilt $ \# G = \# M $,
		falls die Menge $ M $ nichtleer ist.
		
		Damit müssen wir lediglich die Mächtigkeit der Menge $ H^1(M,\Z_2) $
		berechnen um die Anzahl aller nicht isomorphen Spinstrukturen zu kennen. Diese ist stets endlich, da endliche dimensionale Mannigfaltigkeit stets eine endliche $ CW $-Darstellung besitzen.
		
		Es gilt nun für eine $ 1 $-zusammenhängende spin Mannigfaltigkeit.
		\begin{gather*}
			H^1(M,\Z_2)\simeq \Hom(\underbrace{\pi_1(M)}_{=0},\Z_2)=0 
		\end{gather*}
		Wir werden in \cref{existenzspinflächen} beweisen das Flächen
		stets spin sind. Wir müssen genauso wie vorher nur $ H^1(M,\Z_2) $
		berechnen. Wir werden hierzu eine CW-Zerlegung einer Flächen
		benutzen und mit Sätzen aus der algebraischen Topologie folgt nun
		das $ H^1(M,\Z) \simeq H_1(M,\Z)\simeq\Z^{2\gamma} $ gilt und damit 
		$ \# H^1(M,\Z_2) = \# (\Z_2^{2\gamma})  = 2^{2\gamma}$.
		
%		Wir müssen lediglich $ H^1(M,\Z_2) $ berechnen und die transitive
%		Gruppenwirkung benutzen. Da die Mannigfaltigkeit $ 1 $-zusammenhängend ist, gilt mit dem universellen Koeffiziententheorem
%		$ H^1(M,\Z_2)\simeq \Hom(\underbrace{\pi_1(M)}_{=0},\Z_2)=0 $.
%		Die Menge aller Spinstrukturen ist nichtleer, da die Mannigfaltigkeit spin ist. Aufgrund der Transitivität der Gruppenwirkung aus \cref{anzahlspin} und der trivialen Gruppe 
%		folgt nun das es genau eine Spinstruktur bis auf Isomorphie gibt.
%		
%		Wir werden in \cref{existenzspinflächen} beweisen das Flächen
%		stets spin sind. Wir müssen genauso wie vorher nur $ H^1(M,\Z_2) $
%		berechnen. Wir werden hierzu eine CW-Zerlegung einer Flächen
%		benutzen und mit Sätzen aus der algebraischen Topologie folgt nun
%		das $ H^1(M,\Z) \simeq H_1(M,\Z)\simeq\Z^{2\gamma} $ gilt und damit 
%		$ \# H^1(M,\Z_2) = \#(\Z_2^{2\gamma})  = 2^{2\gamma}$.\ŧodo{genauer?}
	\end{proof}
\end{Kor}

\begin{Bsp}
  \begin{enumerate}[(1)]
  \item Das erste Beispiel ist das Triviale. Für die \RMF
    $(\R^n, \scdots_{\R^n})$ ist
    $(\R^n \times \spin_n,\id \times \Ad)$ eine Spinstruktur auf
    dieser riemannischen Mannigfaltigkeit.\footnote{Dies ist auch die
      einzigste Spinstruktur für den $ \R^n $ bis auf
      Isomorphie}. Allgemein gilt das falls $ \Ph{\so_n}{M} $ für eine Mannigfaltigkeit
    $ M^n $ trivial ist, also isomorph zu $ M\times \so_n $ ist,
    dann ist das triviale $ \spin_n $-Haupfaserbündel
    $ (M\times \spin_n,\pi_1,M,\spin_n) $ eine Spinstruktur für $ M $.
    Doch wie wir im nächsten Beispiel sehen werden folgt daraus noch
    nicht das dies dann die einzige Spinstruktur sein muss (bis auf
    Isomorphie).
    \begin{proof}
      Das Tupel $ (\R^n\times \spin_n,\pi_1,\R^n,\spin_n) $ ist ein
      triviales Hauptfaserbündel mit Strukturgruppe $ \spin_n $.  Sei
      ein $ (p,u)\in \R^n\times \spin_n $ gegeben, brachte dann
      folgendes Diagramm
			
%			\begin{center}
%				\begin{tikzcd}
%				\spin_n \arrow[rr, "a \mapsto (p,au)"] \arrow[dd,"\Ad"] && \R^n\times \spin_n \arrow[dd] \aarow  & \\
%				&&& M \\
%				\so_n && \R^n &
%			\end{tikzcd}
%			\end{center}	
      \begin{center}
					
        \begin{tikzcd}
          \spin_n \ar[rr, "a \mapsto \pair{p}{au}"] \ar[dd, "\Ad"'] & &  \R^n\times \spin_n  \arrow[rd] \arrow[dd, "\id\times\Ad"] & \\
          &&& \R^n \\
          \so_n \arrow[rr , "A \mapsto pA"] & & \R^n\times\so_n  \arrow[ru] &
        \end{tikzcd}
      \end{center}

    \end{proof}
		
  \item Ein erstes nichttriviales Beispiel sind zwei nicht isomorphie
    Spinstrukturen der $\S^1$. Diese sind $(\S^1 , \eta_0)$ und
    $(\S^1 \times \Z_2 = \S^1 \times \spin_1 , \eta_1)$, wobei die
    Abbildungen durch $\eta_0 : \S^1 \rightarrow \S^1 , z \mapsto z^2$
    und
    $\eta_1 : \S^1 \times \Z_2 \rightarrow \S^1 , (z,\alpha) \mapsto
    z$
    gegeben sind. Die beiden Spinstrukturen können nicht isomorph
    sein, da die Totalräume verschieden viele Zusammenhangskomponenten
    haben und somit schon als topologische Räume nicht homöomorph sein
    können. Nun sind dies auch alle Repräsentaten von
    Spinstrukturen bis auf Isomorphie auf der $ \S^1 $.
  \item Die $(\CP^{2n},g)$ ist ein Beispiel für eine Familie von
    \RMFen die nicht spin sind, wobei hier $g$ eine riemannische
    Metrik auf $\CP^{2n}$ ist.  \todo{Quelle einfügen}
    
  \item \label{SpinstrSphäre}Ein wichtiges Beispiel für eine \RMF mit deren Spinstruktur
    sind die $ n $-Sphären $\S^n$ für $ n \geq 2 $. Diese besitzen  
    aufgrund von \cref{anzahlspin} genau eine Spinstruktur. Um 
    dieses anzugeben betrachten wir zunächst das orientierte Reperebündel $ \so(\S^n) $. Für ein $ p\in\S^n $ ist der Tangentialraum $ \tang{p}{\S^n} $ mit $ \S^n\subset\R^{n+1} $ 
    gleich zu $ p^\bot\subset\R^{n+1} $. Für ein Element von $ (p,A)\in \Ph{\so_n}{\S^n} $ gilt nun das $ A \in\so_n$ ist und damit folgt
    nun leicht $ \Ph{\so_n}{\S^n} \simeq \so_{n+1} $. Damit folgt
    aufgrund von $ n\geq 2 $ als einzige doppelte Überlagerung der
    $ \so_{n+1} $ die $ \spin_{n+1} $, da diese Überlagerung universell ist. Somit ist $ (\spin_{n+1},\Ad) $ die eindeutige
    Spinstruktur auf der $ \S^n $.
  \item Ein wichtiges Beispiel sind die Spinstrukturen auf dem $ 2 $-Tours $ \TT^2 \coloneqq \S^1\times\S^1$. Diese lassen sich auf sehr
  kompakte Art und Weise angeben.
  Sei hierzu $ \abb{\chi}{\Z^2}{\Z_2}$\footnote{Mit $ \pi_1(\TT^2)=\Z^2 $ ist dies genau wie in \cref{anzahlspin}} ein Gruppenhomomorphismus gegeben
  und betrachte folgende Gruppenwirkungen, wobei $ \Z_2\simeq\sset{+1,-1}\simeq\spin_1 \subset\spin_2 $ benutzt wird.
  \DefMap{\rho_1 : \Z^2\times\R^2}{\R^2}{((n,m),(x,y))}{(x+n,y+m)}
  \DefMap{\rho_2 : \Z_2\times\spin_2}{\spin_2}{(\alpha,z)}{\alpha z}


  Dann sind alle Spinstrukturen gegeben durch  
  \begin{gather*}
  	\R^2 \times \spin_2 / (\rho_1(a,b)(x,y),z) \sim ((x,y),\rho(\chi(a,b))z).
  \end{gather*}
  \begin{smallproof}(diese Spinstrukturen sind nicht isomorph)
  	\todo{wähle flachen zusammenhang und betrachte weg und dessen Parallaeltransport}
  \end{smallproof}
  
	\end{enumerate}
\end{Bsp}

In \cite{KS96} wird für Flächen eine äquivalente Definition
von Spinstruktur angegeben. Diese begründet die Aussage das
eine Spinstruktur die \enquote{Wurzel} aus dem Tangentialbündel ist.

\begin{Def}\label{DefSpin_KS}
	Sei $ (M,g) $ eine Fläche, dann heißt $ (S,\mu) $ eine
	\textit{Spinstruktur} auf $ M $, falls $ S $ ein komplexes Geradenbündel
	über $ M $ ist und die Bündelabbildung $ \abb{\mu}{S}{\T M} $ quadratisch	ist, d.h. $ \mu(\lambda s) = \lambda^2 \mu(s) $ für alle Schnitte
	von $ S $.
\end{Def}

Zeige nun das diese beiden Definitionen sich ineinander überführen
lassen und somit dasgleiche liefern.

\begin{Satz}
	Zu jeder Spinstruktur im Sinne von \cref{DefSpin} existiert
	genau eine Spinstruktur im Sinne von \cref{DefSpin_KS}.
	\begin{proof}
		Sei eine Spinstruktur $ (\Ph{\spin_2}{M},\eta) $ gegeben.
		Mit den Fakt $ \S^1\simeq\so_2\simeq\spin_2 $ und der Abbildung
		$ \abb{\Ad}{\spin_2}{\so_2},z\mapsto z^2 $ liefert uns dies
		die induzierte Abbildung $ \Ph{\spin_2}{M}\rightarrow \Ph{\so_2}{M} $. Betrachten wir nun die davon assoziierten
		Bündel bezüglich der Darstellung $ \abb{\rho}{\S^1}{\End(\c)}, z\mapsto (w \mapsto zw) $, dann erhalten wir die doppelte
		Überlagerung $ S\coloneqq \Ph{\spin_2}{M}\times_{\S^1}\c \rightarrow {\T M = \Ph{\so_2}{M}\times_{\S^1}\c} $. Diese Abbildung ist wohldefiniert, da die Gruppenwirkung der $ \S^1 $ mit der
	    der Abbildung $ z\mapsto z^2 $ vertauscht. Damit erhalten
	    wir die gewünschte Spinstruktur.
	    
	    Sei nun $ (S,\mu) $ gegeben. \todo{andere richtung, entweder mu abbildung nutzen um Pspin zu definieren, oder aus dem ass bündel die spinstr zurückgewinnen}
	    
		
	\end{proof}
\end{Satz}



%% Local Variables:
%% mode: latex
%% TeX-master: "main"
%% End:
