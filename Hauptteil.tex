%!TEX root = main.tex
% vim: tw=0 noet sts=8 sw=8



\section{Zusammenhang von Immersionen und Spinstrukturen}

Im folgenden ist $(M,g)$ eine geschlossene, kompakte, orientierte \mfg.
In der Notation wird die Metrik $g$ unterdrückt.


Um einen Zusammenhang zwischen den Immersionen und den Spinstrukturen
herzustellen ist es notwendig den Raum der Immersionen zu vereinfachen.
Wichtig ist hierbei das wir nur an den Zusammenhangskomponenten interessiert
sind, also können wir den Raum beliebig homotopieren. Das folgende Theorem
liefert uns eine erste Vereinfachung des Raumes.
%satz von smale hirsch als black box, aus intro to the h-principle
%finde gute referenz für dieses Theorem
\begin{Thm}[Smale-Hirsch-Theorem]
	Seien $M^m,N^n$ \mfgen, $m<n$ und M sei geschlossen. Dann ist die Abbildung
	$\abb{\theta}{\Imm{M}{N}}{\Immf{M}{N}}, f \mapsto (f,\mathrm{d} f)$ eine Homotophieäquivalenz.
\end{Thm}

Der Raum von formalen Immersionen lässt sich noch weiter vereinfachen, denn wenn
wir anstatt von nur formalen Immersionen auch diese betrachte bei denen
die Abbildungen zwischen den Tangentialräumen auch noch eine Isometrie ist,
erhalten wir folgendes Resultat.


\begin{Satz}
	Seien die isometrischen, formalen Immersionen gegeben durch
	\begin{gather*}
	\Immfi{M}{\R^3} \coloneqq \set{(f,F) \in \Immf{M}{\R^3}}{F \text{ ist isometrisch }}
	\end{gather*}
	Dann gilt das $\Immf{M}{\R^3}$ und $\Immfi{M}{\R^3}$ homotopieäquivalent sind.
	\begin{proof}
		Betrachte für $t \in [0,1]$ die Abbildung $\abb{\psi_t}{\Immf{M}{\R^3}}{\Immf{M}{\R^3}}$ mit 
		$\psi : (f,F) \mapsto (f,F \circ (F^{\ast}F)^{-\frac{t}{2}})$.
		Dann liefert dies die gewünschte Homotopieäquivalenz.
		\todo{warum liefert sie das, genauer.}
	\end{proof}
\end{Satz}

Für den Tangentialraum des $\R^n$ gilt:
\begin{align*}
	\T \R^n &\simeq \R^n \times \R^n \\
	(p,v) &\mapsto (p,v)
\end{align*}
Damit gilt nun folgendes Resultat

\begin{Lem}
	Die folgende Abbildung ist eine Bijektion
\todo{bijektion oder homotopieäquivelante räume?}
	\begin{align*}
		\Immfi{M}{\R^3} &\rightarrow \set{ (\abb{F_p}{\T_pM}{\R^3})_{p\in M} }{\text{ isometrisch}}\\
		()
	\end{align*}
\end{Lem}


%lemma darüber das eine kompakte, orientierte 2dim riem mfg stets spin ist
\begin{Satz}
	Jede geschlossene, orientierte, \RMF  $(M,g)$ der Dimension $2$ ist spin.
	\begin{proof}
		%todo: beweis
	\end{proof}
\end{Satz}
% satz das die imm und form-imm homotop sind

Sei $\S M$ definiert als $\set{v \in \T M}{ \norm{v} = 1}$, dann
gilt folgender Satz

\begin{Satz}
	Es existiert eine Bijektion
\begin{center}
\begin{tikzcd}
	\set{ (\abb{F_p}{\T_pM}{\R^3})_{p\in M} }{\text{ isometrisch}} \arrow[rr,"\vartheta"] && \set{ \abb{H}{\S M}{\so{3}}}{\S^1 \text{-invariant}} \\
\end{tikzcd}
\end{center}


\begin{proof}
	Sei eine Familie $(\abb{F_p}{\T_pM}{\R^3})_{p\in M}$ von 
	isometrischen Abbildungen gegeben.
	Da es sich bei $M$ um eine zweidimensionale, orientierte riemannische Mannigfaltigkeit handelt, gilt das zu jedem normierten Tangentialvektor $v \in \S_p M$ genau einen Tangentialvektor $J(v) \in \S_p M$
	existiert, sodass $(v,J(v))$ eine positive, orientierte Orthonormalbasis für	$\tang{p}{M}$ ist. Betrachte nun die Abbildung $\abb{H}{\S_p%
		M}{\so{3}}, v \mapsto (F_p(v),F_p(J(v)),F_p(v)\times F_p(J(v)))$. 
	Diese Abbildung ist wohldefiniert aufgrund der Eigenschaften
	der Basis $(v,J(v))$ und dass das Kreuzprodukt der beiden
	Basiselemente in Matrixform ein Element der $\so{3}$ liefert.
	Die $\S^1$ Invarianz dieser Abbildung folgt direkt
	aus der Isometrie der Abbildungen $F_p$.
	\todo{genauer diesen Teil}
	
	Zeige nun das es sich bei dieser Abbildung um eine Bijektion handelt. 
	\begin{description}
		\item[Injektivität:] Seien zwei Familien $(F_p)_p,(G_p)_p$ von
		isometrischen Abbildungen gegeben und sei $\vartheta((F_p)_p)=\vartheta((G_p)_p)$. Dann gilt aufgrund
		der Abbildungsvorschrift von $\vartheta$ das $F_p(v)=G_p(v)$
		für alle $v \in \S_p M$ gilt und somit die Abbildungen gleich
		sind.
		\item[Surjektivität:] Sei eine $\S^1$-invariante Abbildung
		$\abb{H}{\S M}{\so{3}}$ gegeben. Für einen normierten
		Tangentialvektor $v \in \S_p M$ gilt das $H_p(v)$ ein
		Element der $\so{3}$ ist. Betrachte nun die erste Spalte
		der Matrix $w(v)=(H_p(v)_{i,1})_{i=1,2,3}$. Definiere nun eine
		Abbildung $\abb{F_p}{\tang{p}{M}}{\R^3}$ durch 
		\todo{frac und norm vertragen sich nicht}
		$v \mapsto \begin{cases}
		w(\frac{v}{\norm{v}}) & v \neq 0 \\
		0 & v=0.
		\end{cases}$
		
		Diese Abbildung liefert uns nun das gewünschte Urbild zur
		anfangs gewählten Abbildung $H$. 
	\end{description}
	Der Nachweis der Bijektion liefert die Behauptung.
\end{proof}

\end{Satz}

Für den schlussendlichen Satz dieser Arbeit benötigen wir
noch einige Homotopiegruppen der $\so{3}$ und $\spin{3}$.
\begin{Lem}\label{hgroups}
	Für die beiden Liegruppen $\spin{3}$ und $\so{3}$ gelten
	folgende Isomorphien als Mannigfaltigkeiten:
	\begin{align*}
	\spin{3} &\simeq \su(2) \simeq \S^3 \\
	\so{3} &\simeq \RP^3
	\end{align*}
	Damit gelten auch folgende Isomorphien der Homotopiegruppen
	der beide Liegruppen.
	\begin{align*}
	\pi_k(\spin{3}) &\simeq \begin{cases}
	1 & ,k=0,1,2 \\
	\Z & ,k=3
	\end{cases}
	&{\pi_k(\so{3}) \simeq \begin{cases}
		1 & ,k=0,2 \\
		\Z_2 & ,k=1 \\
		\Z & ,k=3
		\end{cases}}
	\end{align*}
\end{Lem}

Nach dieser weiteren Vereinfachung des Raums der Immersionen folgt
nun der letzte Schritt. Wir können nun folgenden Satz zeigen.


%% Local Variables:
%% mode: latex
%% TeX-master: "main"
%% End:
